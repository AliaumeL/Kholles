\exercice[Croissance des fonctions]

On dit que $f : I \to \mb{R}$ est croissante 
quand elle vérifie la propriété suivante~:

\begin{equation}
    \forall (x,y) \in I^2, x \leq y \implies f(x) \leq f(y)
\end{equation}

On dit qu'une fonction $f : I \to \mb{R}$ est \emph{strictement}
croissante quand elle vérifie la propriété de croissance avec 
des inégalités strictes.

\begin{enumerate}
    \item Que dire si l'on remplace l'implication par une équivalence 
        dans la définition de croissance ?
    \item Que dire si l'on remplace l'implication par une équivalence 
        dans la définition de la croissance stricte ?
\end{enumerate}




