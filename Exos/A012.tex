\exercice[Sur les inclusions d'ensembles]

On pose $P = \{ 2k ~|~ k \in \mathbb{N} \}$ et $I = \{ 2k + 1 ~|~ k \in
\mathbb{N} \}$.

\begin{enumerate}
    \item Montrer par récurrence sur $n \in \mathbb{N}$ la propriété suivante~:
        \begin{equation}
            \forall n \in \mathbb{N}, \exists k \in \mathbb{N}, (n = 2k) \vee 
            (n = 2k + 1)
        \end{equation}
    \item En déduire une inclusion entre $P \cup I$ et $\mathbb{N}$
    \item Montrer l'inclusion réciproque
    \item En déduire que $P \cup I = \mathbb{N}$
    \item Montrer que $P \cap I = \emptyset$
    \item On dit qu'un entier est pair quand il est dans $P$ et qu'un 
        entier est impair quand il est dans $I$. Qu'avons nous montré 
        dans cet exercice ?
\end{enumerate}






