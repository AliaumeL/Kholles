\exercice[Fonctions indicatrices]

Soit $E$ un ensemble. Si $A \subseteq E$ on pose 
$\mathds{1}_A (x) = 0$ quand $x \not \in A$ et $\mathds{1}_A (x) = 1$ 
quand $x \in A$.

\begin{enumerate}
    \item Justifier que si $A$ est une partie de $E$
        $\mathds{1}_A$ est bien une fonction de $E$ dans $\{ 0, 1 \}$

    \item Étant donné une fonction $f$ de $E$ dans $\{ 0, 1 \}$
        construire un ensemble $A$ tel que $f = \mathds{1}_A$

    \item En déduire une bijection entre $\mathcal{P}(E)$
        et $E \to \{ 0, 1 \}$

    \item Justifier que si $f_1$ et $f_2$ sont des 
        fonctions de $E$ dans $\{0,1\}$ alors 
        les opérations suivantes définissent bien 
        une fonction de $E$ dans $\{ 0, 1\}$:


        \begin{multicols}{2}
            \begin{enumerate}
                \item $\max (f_1, f_2)$
                \item $\min (f_1, f_2)$
                \item $f_1 \times f_1$
                \item $1 - f_1$
                \item $f_1 + f_2 - f_1 \times f_2$
                \item $(f_1 - f_2)^2$
            \end{enumerate}
        \end{multicols}


    \item En utilisant la bijection, étudier 
        le sens de ces opérations dans $\mathcal{P}(E)$
\end{enumerate}

