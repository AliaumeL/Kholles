\exercice[Ensembles dénombrables]

\begin{enumerate}

\item Expliciter une bijection $f$ entre $\mb{N}$ et $\mb{Z}$.

\item Soit $\fonction{g}{\mb{N}^2}{\mb{N}}{(i,j)}{ i + \sum_{k=0}^{i+j} k}$.

\begin{enumerate}

\item Calculer $g(i,j)$ pour les $i,j$ tels que $i+j \le 3$, puis représenter les valeurs obtenues sur le plan $\mb{N}^2$. Décrire par un dessin le comportement général de $g$.

\item Montrer que $g$ est surjective.

\emph{Pour trouver un antécédent de $n$, on pourra considérer $p := \min \left\{ p \in \mb{N}~|~ \sum_{k=0}^{p} k \le n\right\}$}.

\item Montrer que $g$ est injective.

\end{enumerate}

\item En remarquant que $\forall n \ge 1$, $\mb{N}^{n+1} = \mb{N}^n \times \mb{N}$, construire par récurrence sur $n > 0$ une bijection entre $\mb{N}$ et $\mb{N}^n$.

\end{enumerate}




