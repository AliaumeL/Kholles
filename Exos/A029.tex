\exercice[Propriété universelle du produit]

\begin{enumerate}

\item Si $A$ et $B$ sont deux ensembles finis, quel est le cardinal de $A \times B$ ?

\item Soient $A$ et $B$ deux ensembles non vides.

On pose $\fonction{\pi_A}{A\times B}{A}{(a,b)}{a}$ et $\fonction{\pi_B}{A\times B}{A}{(a,b)}{b}$.

Montrer que $\pi_A$ et $\pi_B$ sont surjectives. Quand a-t-on injectivité de $\pi_A$ ?

\item Soit $C$ un ensemble et deux fonctions $f_A : C \rightarrow A$ et $f_B : C \rightarrow B$. Montrer qu'il existe une unique fonction $h : C \rightarrow A \times B$ telle que $\pi_A \circ h = f_A$ et $\pi_B \circ h = f_B$. 

\emph{On pourra raisonner par analyse-synthèse.}

\item Faire un dessin de la construction. On représentera les fonctions $\pi_A, \pi_B, f_A, f_B$ et $h$ par des flèches entre les ensembles $A, B, A \times B$ et $C$.

\end{enumerate}



