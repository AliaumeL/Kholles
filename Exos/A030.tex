\exercice[Propriété universelle du coproduit]

 Si $A$ et $B$ sont deux ensembles, on définit l'union disjointe $A \uplus B := \{0\} \times A \cup \{1\} \times B $

\begin{enumerate}

\item Si $A$ et $B$ sont deux ensembles finis, quel est le cardinal de $A \uplus B$ ? Celui de $A \times B$ ?
\item Soient $A$ et $B$ deux ensembles.

On pose $\fonction{\mu_A}{A}{A\uplus B}{a}{(0,a)}$ et $\fonction{\pi_B}{B}{A\uplus B}{b}{(1,b)}$.

Montrer que $\mu_A$ et $\mu_B$ sont injectives. Quand a-t-on surjectivité de $\mu_A$ ?

\item Soit $C$ un ensemble, et deux fonctions $f_A : A \rightarrow C$ et $f_B : B \rightarrow C$. Montrer qu'il existe une unique fonction $h : A \uplus B \rightarrow C$ telle que $h \circ \mu_A  = f_A$ et $h \circ \mu_B  = f_B$. 

\emph{On pourra raisonner par analyse-synthèse.}

\item Faire un dessin de la construction. On représentera les fonctions $\mu_A, \mu_B, f_A, f_B$ et $h$ par des flèches entre les ensembles $A, B, A \times B$ et $C$.

\end{enumerate}


