\exercice[Propriété universelle du quotient]

Soit $A$ un ensemble et $\sim$ une relation d'équivalence sur $A$.

\begin{enumerate}

\item Montrer que les classes d'équivalence de $\sim$ forment une partition $(A_i)_{i \in I}$ de $A$. Dans la suite, on note $\overline{A} = \{A_i~|~i \in I\}$ l'ensemble des classes.

\item Soit $\fonction{\pi}{A}{\overline{A}}{x}{A_i \text{ tel que } x \in A_i}$.

\begin{enumerate}

\item Représenter (patates) $\overline{A}$ et $\pi$ dans le cas où $A = \{0,1,2\}$ avec $0 \sim 2$ et $1 \not \sim 2$.

\item Dans le cas général, pourquoi $\pi$ est-elle bien définie ?

\item Montrer que $\pi$ est surjective.

\end{enumerate}

\item Soit $B$ un ensemble et $f : A\rightarrow B$ telle que $\forall x,y \in A$, $x \sim y \Rightarrow f(x) = f(y)$.

Montrer qu'il existe une unique fonction $\overline{f} : \overline{A} \rightarrow B$ telle que $f = \overline{f} \circ \pi$.

\emph{On pourra raisonner par analyse-synthèse.}

\item Faire un dessin de la construction. On représentera les fonctions $\pi, f$ et $ \overline{f}$ par des flèches entre les ensembles $A, \overline{A}$ et $B$.
\end{enumerate}



