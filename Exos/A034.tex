\exercice[Vision graphique]

Soit $f : E \to F$ une \emph{fonction}.

\begin{itemize}
    \item Dans le cas où $E = F = \mb{R}$ 
        définir le graphe de $f$ noté $\Gamma_f$ 
        comme un sous ensemble de $\mb{R}^2$.

    \item Généraliser cette définition 
        au cas où $E$ et $F$ sont quelconques 

    \item Montrer que si $f$ est une fonction 
        $\Gamma_f$ est une relation de $E \times F$ 
        qui vérifie
        \begin{enumerate}[(i)]
            \item $\forall x \in E, \exists y \in F, x \Gamma_f y$
            \item $\forall x \in E, \forall (y_1,y_2) \in F^2, 
                x \Gamma_f y_1 \wedge x \Gamma_f y_2 \implies 
                y_1 = y_2$
        \end{enumerate}

    \item Si $R$ est une relation de $E \times F$
        on pose $R^{op}$ une relation de $F \times E$
        définie par $\{ (y,x) \in E\times F ~|~ x R y \}$

        Montrer que si $f$ est bijective, $\Gamma_f^{op} = \Gamma_{f^{-1}}$.

    \item En général si $f$ est une fonction, quelle 
        à quelle propriété de $\Gamma_f^{op}$ correspond la 
        \emph{surjectivité} ? \emph{L'injectivité} ? On 
        se servira des propriétés $(i)$ et $(ii)$ définies 
        sur les relations.
\end{itemize}


