\exercice[Symétrie des bornes]

Soit $(E, \le)$ un ensemble ordonné. On suppose que toute partie non vide et majorée de $E$ admet une borne supérieure.

\begin{enumerate}

\item Soit $A \neq \varnothing$ une partie minorée de $E$. Montrer que $B := \{x~|~ \forall y \in A, x \le y\}$  admet une borne supérieure.

\item On va montrer que $A$ admet une borne inférieure. 
\begin{enumerate}

\item Montrer que $\sup(B)$ est un minorant de $A$. \emph{On remarquera que $A$ est contenu dans l'ensemble des majorants de $B$}.

\item Montrer que $\sup(B)$ est le plus petit d'entre eux et conclure.

\end{enumerate}

\item On suppose maintenant que $E$ possède un plus grand et un plus petit élément. Montrer que toute partie de $E$ admet des bornes supérieure et inférieure.

\end{enumerate}



