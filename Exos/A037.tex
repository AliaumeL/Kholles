\exercice[Prolongement des applications]

Soient $A,B$ deux ensembles et $\mc{E} = \{(F,f)~|~F \subseteq A \text{ et } f : F \rightarrow B\}$.

\begin{enumerate}

\item Soit $\preccurlyeq$ défini par $(F,f) \preccurlyeq (G,g)$ ssi $F \subseteq G$ et $\forall x \in F, f(x) = g(x)$ ($g$ est un prolongement de $f$). Montrer que $\preccurlyeq$ une relation d'ordre sur $\mc{E}$.

\item A quelles conditions sur $A,B$ l'ordre est-il total ?

\item Montrer que $\mc{E}$ possède un plus petit élément.

\item Soit $\mc{F}\subseteq \mc{E}$ un sous-ensemble totalement ordonné de $\mc{E}$. Montrer que $\mc{F}$ est majoré.

\emph{Remarque : on dit que $\mc{E}$ est un ensemble inductif.}

\end{enumerate}





