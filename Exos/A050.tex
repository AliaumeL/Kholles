\exercice[Une version simple de l'inégalité de Jensen]

Soit $f: \mb{R} \mapsto \mb{R}$ une fonction vérifiant pour tous réels $x_1, x_2$ et $t \in [0,1]$
$$f(t x_1 +(1-t) x_2) \le t f(x_1) +(1-t) f(x_2)$$  (une telle fonction est dite convexe).
Montrer que si les $\lambda_1 \dots \lambda_n$ sont des réels positifs ou nuls tels que $\sum_{i=1}^n \lambda_i = 1$, alors pour tous réels $x_1 \dots x_n$ on a :
$$f\left(\sum_{i=1}^n \lambda_i x_i\right) \le \sum_{i=1}^n \lambda_i f(x_i).$$


