\exercice[Binet, Lagrange, Cauchy-Schwarz]

Soient $x_1 \dots x_n, y_1 \dots y_n, a_1 \dots a_n, b_1 \dots, b_n \in \mb{R}$

\begin{enumerate}

\item Développer $\displaystyle \left(\sum_{i=1}^n a_i \right)\left(\sum_{i=1}^n b_i \right)$.

\item Soient $x_1 \dots x_n, y_1 \dots y_n, a_1 \dots a_n, b_1 \dots, b_n \in \mb{R}$. Montrer la formule de Binet :
$$ \left(\sum_{i=1}^n a_i x_i\right)\left(\sum_{i=1}^n b_i y_i\right) = \displaystyle \left(\sum_{i=1}^n a_i y_i\right)\left(\sum_{i=1}^n b_i x_i\right) + \sum_{1 \le i < j \le n} (a_i b_j - a_j b_i)(x_i y_j - x_j y_i).$$

\item En déduire l'identité de Lagrange :

$$\displaystyle \left(\sum_{k=1}^n a_k^2\right)\left(\sum_{k=1}^n b_k^2\right) - \left(\sum_{k=1}^n a_k b_k\right)^2 = \sum_{1 \le i < j \le n} (a_i b_j - a_j b_i)^2$$


\item Etablir l'inégalité de Cauchy-Schwarz :
$$\displaystyle \left(\sum_{k=1}^n a_k b_k\right)^2 \le \left(\sum_{k=1}^n a_k^2\right)\left(\sum_{k=1}^n b_k^2\right)$$

\item En déduire que $ \displaystyle \left(\sum_{k=1}^n c_k\right)\left(\sum_{k=1}^n \frac{1}{c_k}\right) \ge n^2$ si $c_1 \dots c_k$ sont des réels strictement positifs. Interpréter ce résultat en termes d'aires.

\end{enumerate}


