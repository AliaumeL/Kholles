\exercice[Interpolation de Lagrange revisitée]

Soient $x_1 \dots x_n \in \mb{R}$ tous distincts. Notons $\displaystyle P_i =  \prod_{\substack{1 \le j \le n \\ j \neq i}} (x_i - x_j)$.

Montrer que $\displaystyle \sum_{i=1}^n \frac{(x_i)^r}{P_j} = 0$ si $0 \le r < n-1$,  $=1$ si $r = n-1$, $=\displaystyle \sum_{i=1}^n x_i$ si $r = n$.















