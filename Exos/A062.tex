\exercice

On rappelle que l'exponentielle complexe est définie par $$\text{e}^{a+\text{i}b} := \text{e}^{a} \text{e}^{\text{i}b} = \text{e}^{a} (\cos(b) + \text{i} \sin(b)).$$

\begin{enumerate}


\item Soit $f: \mb{C} \rightarrow \mb{C}, z \mapsto \frac{\text{e}^{\text{i}z}+\text{e}^{-\text{i}z}}{2}$ et $g: \mb{C} \rightarrow \mb{C}, z \mapsto \frac{\text{e}^{\text{i}z}-\text{e}^{-\text{i}z}}{2\text{i}}$ .

Que valent $f(x)$ et $g(x)$ quand $x \in \mb{R}$ ? De quelles fonctions réelles $f$ et $g$ sont-elles les "extensions" aux nombres complexes ?

\item Montrer que $f(z)^2 + g(z)^2 = 1$. Cela est-il cohérent avec la question précédente ?

\item Prouver la formule de Moivre complexe : $\text{e}^{\text{i}z} = f(z) + \text{i} g(z)$.

\end{enumerate}




