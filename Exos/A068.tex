\exercice[Autour des racines 3-èmes]

\begin{enumerate}

\item On note $\text{j} = \text{e}^{\frac{2 \text{i} \pi}{3}}$. Que valent $j^3$ et $j^4$ ? Les représenter sur le cercle unité.

\item Quelles sont les solutions dans $\mb{C}$ l'équation $1 + z + z^2 = 0$.

\item Soit $f : \mb{C}^3 \rightarrow \mb{C}$ définie par $f(z_1, z_2, z_3) = \left[z_1 + \text{j} z_2 + \text{j}^2 z_3\right]^3$. Montrer que pour tous $z_1, z_2, z_3 \in \mb{C}$ on a $f(z_1, z_2, z_3) = f(z_2, z_3, z_1) = f(z_3, z_1, z_2)$.

\end{enumerate}

