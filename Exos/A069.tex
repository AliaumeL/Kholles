\exercice[Un peu plus de $j$]

On rappelle que $j = e^{2i\pi /3}$.
On considère les trois somme suivantes~:

\begin{equation*}
    A_n = \sum_{k = 0 \wedge k \equiv 0 [3]}^{n} { n \choose k} 
\end{equation*}

\begin{equation*}
    B_n = \sum_{k = 0 \wedge k \equiv 1 [3]}^{n} { n \choose k} 
\end{equation*}

\begin{equation*}
    C_n = \sum_{k = 0 \wedge k \equiv 2 [3]}^{n} { n \choose k} 
\end{equation*}

\begin{enumerate}
    \item Calculer $S_n = \sum_{k = 0}^n { n \choose k } j^k$
    \item Montrer que $S_n = A_n + j B_n + j^2 C_n$
    \item En déduire que $\overline{S_n} = A_n + j^2 B_n + jC_n$
    \item Calculer $A_n + B_n + C_n$
    \item En dédure une expression de $A_n$, $B_n$ et $C_n$
\end{enumerate}










