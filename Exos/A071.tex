\exercice[Fonctions symétriques de 2 variables]

Soient $z, z' \in \mb{C}$ on note $S = z + z'$ et $P = z z'$.

\begin{enumerate}

\item Exprimer $z^2 + z'^2$ en fonction de $P$ et $S$. Que peut-on en déduire si $P$ et $S$ sont réels ?

\item L'écriture précédente est de la forme $\sum_{i=1}^n \lambda_i P^{j_i} S^{j'_i}$ avec $\lambda_i \in \mb{R}$ et $j_i, j'_i \in \mb{N}$. Montrer que $z^2 + z'$ ne peut pas s'écrire de cette manière. On pourra prendre $z = \text{i}$ et $z' = - \text{i}$.

\item Exprimer $z^2 z' + z'^2z $ en fonction de $P$ et $S$. En déduire un expression de $z^3 + z'^3$.

\item Que peut-on dire pour $z^n + z'^n$ ?

\end{enumerate}


