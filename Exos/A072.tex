\exercice[Homothéties dans le plan complexe]

On considère $f$ une fonction qui vérifie la propriété 
suivante~:

\begin{equation*}
    \forall (z,z') \in \mathbb{C}^2,
    \forall (\alpha, \beta) \in \mathbb{R}^2,
    f( \alpha z + \beta z') = \alpha f (z) + \beta f(z')
\end{equation*}

\begin{enumerate}
    \item Montrer que $f(0) = 0$
    \item Montrer que $f$ est entièrement déterminée 
        par $f(1)$ et $f(i)$

    \item On suppose désormais que $f$ vérifie de plus 
        la propriété suivante~: $\forall z \in \mathbb{C},
        \exists \lambda \in \mathbb{R}, f (z) = \lambda z$.
        Montrer que $f$ vérifie alors~: $\exists \lambda
        \in \mathbb{R}, \forall z \in \mathbb{C}, f(z) = \lambda z$.
\end{enumerate}


