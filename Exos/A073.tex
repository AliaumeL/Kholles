\exercice[Transformée de Fourier Rapide]

On note $\alpha_0, \dots, \alpha_{n-1}$ des nombres réels.
On pose $\omega = e^{2i\pi / n}$, et 
pour $0 \leq 1 \leq n - 1$~:

\begin{equation*}
    \beta_i = \sum_{k = 0}^{n-1} \alpha_k (\omega^i)^k
\end{equation*}

\begin{enumerate}
    \item 
        Calculer $\sum_{i = 0}^{n - 1} \beta_i (\omega^{-1})^i$
    \item 
        Calculer $\sum_{i = 0}^{n-1} \beta_i (\omega^{-k})^i$

    \item En déduire qu'il est possible de récupérer les coefficients 
        $\alpha_k$ à partir des coefficients $\beta_k$.
\end{enumerate}


