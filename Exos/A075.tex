\exercice[Inégalité triangulaire généralisée]

\begin{enumerate}

\item Montrer que pour tout $n \ge 1$, pour tous $z_1 \dots z_n \in \mb{C}$, on a $\displaystyle \left|\sum_{k=1}^n z_k\right| \le \sum_{k=1}^n | z_k |$.

On pourra utiliser, sans la redémontrer, l'inégalité triangulaire du cours.

\item Rappeler dans quel cas $|z_1 + z_2| = |z_1| + |z_2|$.

\item Montrer que s'il existe $1 \le i,j \le n$, $|z_i + z_j| < |z_i| + |z_j|$, alors $ \left|\sum_{k=1}^n z_k\right| < \sum_{k=1}^n | z_k |$ 

\item Déduire de 2 et 3 à quelles conditions l'inégalité de la question 1 est une égalité. L'interpréter géométriquement.

\end{enumerate}


