\exercice[Convexité]

On dit que $f : I \to \mb{R}$ est convexe quand elle vérifie 
la propriété suivante

\begin{equation*}
    \forall (x,y) \in I^2, \forall \lambda \in [0,1], 
    f(\lambda x + (1-\lambda)y) \leq \lambda f(x) + (1-\lambda) f(y)
\end{equation*}

Soit $a \in I$, on note $\Delta_a$ la fonction définie comme suit

\begin{equation*}
    \Delta_a (x) = \frac{f(x) - f(a)}{x - a}
\end{equation*}

\begin{enumerate}
    \item Quel est l'ensemble de définitiion de $\Delta_a$ ?
    \item Si $(x,y) \in I^2$ et $\lambda \in [0,1]$ 
        montrer que $\lambda x + (1 - \lambda y) \in I$
    \item Montrer que si pour tout $a \in I$ $\Delta_a$ 
        est une fonction croissante alors $f$ est convexe.
        \emph{(Indication: supposer $x < y$, $\lambda \in ]0,1[$ 
            et remarquer 
            $\lambda x + (1 - \lambda y) < y$
            puis utiliser la croissance de $\Delta_x$)}
    
    \item On suppose maintenant $f$ convexe, et $a \in I$.
        \begin{enumerate}
            \item Si $(x,y) \in I^2$ et $x < y < a$, en considérant 
                $\lambda = (x - a) / (y-a)$ montrer que 
                $\Delta_a (x) \leq \Delta_a (y)$

            \item Si $(x,y) \in I^2$ et $a < x < y$, montrer que 
                $\Delta_a(x) \leq \Delta_a (y)$

            \item Si $(x,y) \in I^2$ et $x < a < y$, montrer 
                que $\Delta_a (x) \leq \Delta_a (y)$. 
                \emph{(Indication: considérer $\lambda = (a - y)/(x-y)$)}
        \end{enumerate}
        En déduire que $\Delta_a$ est croissante.
    \item Conclure que $f$ est convexe si et seulement si $\Delta_a$ est 
        croissante pour tout $a \in I$.
   
    \item En déduire que si $f$ est convexe et dérivable sur $I$, alors $f'$ 
        est croissante sur $I$ \emph{(Indication: utiliser la croissance de $\Delta_x$ 
        pour $x$ entre $a$ et $b$)}.

    \item Que dire de la réciproque ? \emph{(Indication: utiliser 
            $g(\lambda) = \lambda f(x) + (1 - \lambda)f(y) - f (\lambda x +
            (1-\lambda) f (y))$)}

    \item Montrer que si $f$ est deux fois dérivable, $f$ est convexe 
        si et seulement si $f''$ est positive.
\end{enumerate}


