\exercice[Une équation de Ricatti]

On cherche les solutions à valeurs complexes de $y' + y + y^2 + 1 = 0$ $(R)$. On raisonne par analyse-synthèse, soit $f$ une solution potentielle.

\begin{enumerate}

\item  \begin{enumerate}

\item A quelle condition sur $c \in \mb{C}$ la fonction constante $x \mapsto c$ est elle solution ?

\item Si $c \in \mb{C}$ est choisi comme à la question précédente, montrer que $g : t \mapsto f(t) -c$ est solution de $y' + (2\text{j} +1) y + y^2 = 0$ $(E)$

\end{enumerate}
 
\item \begin{enumerate}
\item On suppose que $g$ ne s'annule jamais. Montrer que $h : t \mapsto 1/g(t)$ est solution de $-y' + (2 \text{j}+1)y +1 = 0$.

\item Résoudre cette équation.

\end{enumerate}


\item Conclure quant à la forme des solutions de $(R)$ quand $g$ ne s'annule en aucun point.

\item Que se passait-il si $g$ s'annulait en un point dans l'équation $(E)$ ?

\emph{Indication : on pensera à Cauchy-Lipschitz.}
\end{enumerate}




