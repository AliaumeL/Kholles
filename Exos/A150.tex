\exercice[Quand la norme passe à l'intégrale]

Soit $E$ un espace euclidien de dimension finie et $f : [a,b] \rightarrow E$ une application continue telle que $\displaystyle \left\lVert\int_a^b f(t) \di t \right\lVert = \int_a^b \lVert f(t)\lVert \di t$.

\begin{enumerate}

\item Que dire si $\left\lVert\int_a^b f(t) \di t \right\lVert = 0$ ?

\item On suppose désormais $\neq 0$. Soit $e_1 = \int_a^b f(t) \di t \Big/  \left\lVert\int_a^b f(t) \di t \right\lVert $. On complète $e_1$ en une base orthonormée $e_1 \dots e_n$ de $E$, et on note $f = \sum_i f_i e_i$.
\begin{enumerate}

\item Montrer que $\int_a^b f(t) \di t = (\int_a^b \lVert f(t) \rVert \di t) e_1$ puis que $\int_a^b f(t) \di t = (\int_a^b \lVert f(t) \rVert \di t) e_1$.

\item Utiliser ce résultat pour obtenir $f_i = 0$ pour $i \ge 2$.

\item Conclure que $f = \lVert f \lVert e_1$.

\end{enumerate}

\item Ce résultat est-il toujours vrai avec une norme quelconque ?

\end{enumerate}





