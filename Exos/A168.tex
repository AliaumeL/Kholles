\exercice[Suites convergentes vers les irrationnels]

\begin{enumerate}

\item Soit $(u_n)_{n \ge 0}$ une suite d'entiers convergente. Montrer qu'elle est stationnaire.

\item Soit $x\in \mb{R}^*_+ \smallsetminus \mb{Q}$ Monter qu'il existe une suite $\left(\frac{p_n}{q_n}\right)_{n \ge 0}$ de limite $x$ telle que $p_n, q_n \in \mb{N}^*$.

\item On suppose que $q_n \not \rightarrow + \infty$.

\begin{enumerate}

\item Montrer qu'il existe une sous-suite $(q_{\varphi(n)})_{n \ge 0}$ qui est bornée. En déduire qu'il existe une sous-suite $(q_{\psi(n)})_{n \ge 0}$ stationnaire.

\item Montrer que $\left(\frac{p_{\psi(n)}}{q_{\psi(n)}} \times q_{\psi(n)}\right)_{n \ge 0}$ est stationnaire.

\item Conclure à une contradiction.
\end{enumerate}

\item On sait maintenant que $q_n \to + \infty$. On suppose que $p_n \not \rightarrow + \infty$. Montrer qu'il existe une sous-suite $\left(\frac{p_{\nu(n)}}{q_{\nu(n)}}\right)_{n \ge 0}$ tendant vers $0$, puis conclure à une contradiction.

\end{enumerate}

