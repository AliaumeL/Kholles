\exercice[Lemme de Ces\`aro]

Soit $(u_k)_{k \ge 0}$ une suite réelle convergente vers une limite finie $l$. On souhaite montrer que la suite $\displaystyle S_n := \frac{1}{n+1} \sum_{k=0}^n u_k $ converge également vers $l$.

\begin{enumerate}

\item Montrer que pour tout $n \ge m \ge 0$ on a : 

$$ \left| \left( \frac{1}{n+1} \sum_{k=0}^n u_k \right) - l \right| \le \left(\frac{1}{n+1}  \sum_{k=0}^{m-1} \left| u_k - l \right|\right)+ \left(\frac{1}{n+1} \sum_{k=m}^n \left| u_k - l \right|\right)$$

\item On fixe à partir dans cette question et la suivante $\varepsilon > 0$.

\begin{enumerate}

\item Pourquoi existe-t-il $m \ge 0$ tel que $\forall k \ge m$, $|u_k - l | \le \varepsilon /2$ ?

\item En déduire que pour tout $n \ge m$, $\frac{1}{n+1} \sum_{k=m}^n \left| u_k - l \right| \le \varepsilon /2$.

\end{enumerate}

\item $m$ étant fixé, monter qu'il existe $n_0 \ge m$ tel si $n \ge n_0$, $\frac{1}{n+1}  \sum_{k=0}^{m-1} \left| u_k - l \right| \le \varepsilon /2$.

\item Déduire de ce qui précède la convergence de $(S_n)_{n \ge 0}$.

\item A-t-on la réciproque de ce résultat ?




\end{enumerate}





