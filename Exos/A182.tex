\exercice[Point fixe itéré]

Soit $f : \mb{R} \rightarrow \mb{R}$ une fonction continue et $f^n = f \circ \dots \circ f$ ($n$ fois).
\begin{enumerate}

\item Si $f$ admet un point fixe, est-ce le cas de $f^n$ ?

\item On suppose que $f^n$ possède un unique point fixe. Montrer que $f$ possède un point fixe, et qu'il est unique. \emph{Indication : considérer $f(y)$ si $y$ est le point fixe de $f^n$}.

\item On suppose dans la suite que $f^n$ possède un point fixe $y$, pas nécessairement unique. On va montrer que $f$ possède quand même un point fixe. Soit $\phi : x \mapsto f(x) -x$.

\begin{enumerate}

\item Montrer que $\displaystyle \sum_{k=0}^{n-1} \phi(f^k(y)) = 0$.

\item En déduire que $\phi$ change de signe (au sens large) et conclure.

\end{enumerate}

\item Trouver une fonction $f : \mb{R} \rightarrow \mb{R}$ admettant un unique point fixe, telle que $f^2$ ait strictement plus d'un point fixe.


\end{enumerate}


