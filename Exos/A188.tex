\exercice[Equation fonctionnelle de la moyenne]

On cherche les fonctions continues $f : \mb{R} \rightarrow \mb{R}$ telles que $\forall x,y \in \mb{R}$, $f(\frac{x+y}{2}) = \frac{f(x) +f(y)}{2}$.

\begin{enumerate}

\item Soit $f$ une fonction de la forme précédente. On suppose de plus $f(0) = f(1) = 0$.

\begin{enumerate}

\item Montrer que $f(2x) = 2 f(x)$.

\item Montrer que $f$ est $2$-périodique. \emph{On pensera à utiliser les symétries autour de $0$ et $1$.}

\item En remarquant que $f$ doit être bornée, conclure que $f$ est nulle.

\end{enumerate}

\item Que dire dans le cas général ? \emph{Indication : comment se ramener au cas précédent ?}

\end{enumerate}


