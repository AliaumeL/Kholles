\exercice[Automorphismes du corps réel]

Soit $f : \mb{R} \rightarrow \mb{R}$ telle que $\forall x,y \in \mb{R}$, $f(x+y) = f(x) + f(y)$ et $f(xy) = f(x) f(y)$.

\emph{Attention on ne suppose pas $f$ continue.} 

\begin{enumerate}

\item Montrer que $f(0) = 0$ et $f(1) = \pm 1$. Dans la suite on supposera $f(1) = 1$.

\item Montrer que $f$ est l'identité sur $\mb{N}$, puis sur $\mb{Z}$, et enfin sur $\mb{Q}$.

\item Montrer que $f$ est croissante. \emph{Indication : $f(x^2) = f(x)^2$.}

\item En déduire que $f$ est l'identité. \emph{On pensera aux limites monotones.}

\item Que se passait-il si $f(1) = -1$ ?

\end{enumerate}





