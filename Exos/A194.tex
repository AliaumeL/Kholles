\exercice[Le Grand Classique]

On pose $A = (4444)^{4444}$. On note $S(x)$ la somme des chiffres 
de $x$ en base $10$. On pose $B = S(A)$, $C = S(B)$, $D = S(C)$.

\begin{enumerate}
    \item Montrer que $S(x) \equiv x [9]$
    \item En déduire $A \equiv B \equiv C \equiv D [9]$
    \item Réduire $4444$ modulo $9$
    \item En déduire la valeur de $D$ modulo $9$
    \item Montrer que $D \leq 14$, en déduire la valeur de $D$
\end{enumerate}


