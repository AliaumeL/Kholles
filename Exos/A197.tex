\exercice[Un théorème de Liouville]

Soit $p$ un entier, on veut résoudre en $m \in \mb{N}^*$ l'équation suivante : $(p-1)! +1 = p^m$.

\begin{enumerate}

\item Traiter les cas $p = 1,2,3,4,5$.

\item Désormais $p > 5$ et on suppose qu'il existe $m$ tel que $(p-1)! +1 = p^m$.

\begin{enumerate}

\item Montrer que $(p-1)^2 | (p-1) !$.

\emph{Indication : montrer d'abord que $p$ est impair, et donc que $\frac{p-1}{2}$ est entier.}

\item En utilisant une identité bien choisie, montrer que $(p-1) | 1+ p + \dots + p^{m-1}$.

\item En déduire que $(p-1) | m$, puis que $m \ge p-1$.

\item Montrer qu'alors $p^m > (p-1)! +1$. Etait-ce possible ?

\end{enumerate}

\end{enumerate}




