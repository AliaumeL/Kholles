\exercice[Nombres de Mersenne]

\begin{enumerate}

\item 

\begin{enumerate}

\item Montrer que si $\alpha \ge 2$ entier et $m \in \mb{N}$, $\alpha-1 | \alpha^m -1 $.

\item Si $\alpha^m -1$ est premier, montrer que $\alpha = 2$. En déduire qu'alors $m$ est premier.

\end{enumerate}

\item Calculer $2^{11} [23]$. La réciproque de la question précédente est-elle vraie ?

\item On définit la suite $M_n := 2^n -1$ pour $n \ge 0$.

\begin{enumerate}


\item Montrer que $M_{m+n} = 2^n M_{m} + M_n$. En déduire que $M_{m+n} \wedge M_n = M_{m} \wedge M_n$, puis que $M_m \wedge M_n = M_n \wedge M_r $ où $r$ reste de la division euclidienne de $m$ par $n$.

\item Conclure que $M_m \wedge M_n = M_{n \wedge m}$. \emph{On pensera à l'algorithme d'Euclide.}
\end{enumerate}

\end{enumerate}


