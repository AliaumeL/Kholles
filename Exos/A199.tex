\exercice[Nombres de Fibonacci]

On définit la suite $(F_n)_{n \ge 0}$ par $F_0 = 0 $, $F_1 =1$ et $F_{n+2} = F_{n+1} + F_n$.

\begin{enumerate}

    \item \emph{(Optionnel)} \begin{enumerate}\item Montrer que $\forall n \ge 1$, $F_{n+1} F_{n-1} - (F_n)^2 = (-1)^n$.

        \item En déduire $F_n \wedge F_{n+1} = 1$. \emph{(Soit en utilisant la
                question précédente, soit en procédant directement 
                par récurrence car $F_{n+1} = F_n + F_{n-1}$ ... division
                euclidienne...)}
\end{enumerate}

\item \begin{enumerate}

\item Montrer que $\forall n \ge 0$, $\forall m \ge 1$, $F_{m+n} = F_m F_{n+1} + F_{m-1} F_n$.

\item En déduire que $F_n \wedge F_{m+n} = F_n \wedge F_m$, puis $F_n \wedge F_{m} = F_n \wedge F_r$ où $r$ est le reste de la division euclidienne de $m$ par $n$.

\item Conclure que $F_n \wedge F_{m} = F_{m \wedge n}$. \emph{On pensera à l'algorithme d'Euclide.}

\end{enumerate}

\end{enumerate}




