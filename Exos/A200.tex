\exercice[Nombres de Fermat]

\begin{enumerate}

\item \begin{enumerate}

\item Montrer que si $\alpha \ge 2$ entier, et $m$ est impair alors $\alpha +1 | \alpha^m +1$

\emph{Indication : remarquer que $\alpha^m +1 = \alpha^m - (-1)^m$.}

\item Montrer que si $m$ a un facteur impair, alors $2^m+1$ n'est pas un nombre premier. Comment s'écrivent  les entiers $m$ sans facteurs impairs ?

\end{enumerate}

\item Pour $n \ge 0$, on appelle $n$-ième nombre de Fermat noté $F_n$ l'entier $2^{2^n}+1$.

\begin{enumerate}

\item Calculer $F_0$, $F_1$, $F_2$, $F_3$ et justifier qu'ils sont premiers.

\emph{Remarque : $F_4$ aussi est premier, mais pas $F_5 = 641 \times 6700417$.}

\item Montrer que pour $n < m$, $F_n$ et $F_m$ sont premiers entre eux.

\emph{Indication : écrire $m = n+k$ puis faire apparaître $F_n$ dans $F_{n+k} = F_m$.}

\item En déduire qu'il existe une infinité de nombres premiers.

\end{enumerate}



\end{enumerate}






