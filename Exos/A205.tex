\exercice[Théorème de Wilson sans expliciter $\mb{F}_p$]

On souhaite montrer que $p \ge 2$ est premier si et seulement si $(p-1)! \equiv -1 [p]$ $(*)$.

\begin{enumerate}

\item On suppose que $p$ vérifie $(*)$. Considérons $0 < a <p$ tel que $a | p$. Montrer que $a =1$. 

\emph{Indication : montrer que $a | 1$}. Conclure quant à une implication du théorème.

\item Vérifier explicitement $(*)$ lorsque $p = 2,3,5$.

\item On suppose dans la suite $p$ premier $> 3$ et on note $A = \{2 \dots p-2\}$.

\begin{enumerate}

\item Montrer que pour $i \in A$, il existe un unique entier $f(i) \in A$ tel que $if(i) \equiv 1 [p]$. \emph{Indication : remarquer que $i$ et $p$ sont premiers entre eux.}

\item Montrer que $f(f(i)) =i$ et $f(i) \neq i$.

En regroupant les termes 2 par 2, en déduire que $\prod_{i \in A} i \equiv 1 [p]$.

\item Conclure.

\end{enumerate}

\end{enumerate}


