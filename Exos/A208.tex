\exercice[Répartition des nombres premiers]

On note $v_p(n)$ l'exposant de $p$ dans la décomposition de $n$ 
en facteurs premiers. On note $\pi(x)$ le nombre de nombres premiers 
inférieurs à $x$.

\begin{enumerate}
    \item Montrer l'égalité suivante
        \begin{equation*}
            v_p (n!) = \sum_{k = 1}^{+\infty} \left\lfloor \frac{n}{p^k} \right\rfloor
        \end{equation*}

    \item Montrer que $v_p (a b) = v_p (a) + v_p (b)$. En déduire que $v_p (a^2)
        = 2 v_p (a)$. Montrer un résultat similaire pour la division.

    \item Montrer que ${ 2n \choose n }$ divise le produit suivant~:

            \begin{equation*}
                \prod_{p \in \mb{P}, p \leq 2n} p^{ \lfloor \frac{\ln 2n}{\ln p}
                \rfloor}
            \end{equation*}

            \emph{Indication: utiliser les questions précédentes sur le coefficient
            binomial pour déterminer ses diviseurs premiers.}

    \item Montrer que ${ 2n \choose n } \leq (2n)^{\pi (2n)}$


    \item Montrer que quand $x \to +\infty$
        \begin{equation*}
            \frac{x}{\ln x} = O (\pi (x)) 
        \end{equation*}
\end{enumerate}




