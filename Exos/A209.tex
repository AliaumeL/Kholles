\exercice[Existence d'un idempotent]

Soit $(E, \times)$ un magma associatif fini et $a \in E$.

\begin{enumerate}

\item Pourquoi l'expression $a^k := a \times a \times \cdots \times a$ a-t-elle un sens ?

\item \begin{enumerate} \item Montrer qu'il existe $n \ge 0 $ et $m > 0$ tel que $a^{2^n} = a^{2^{n+m}}$.

\item Soit $b := a^{2^{n}}$. Montrer que $b^{2^{m}} = b$. 

\item En déduire que $s := b^{2^{m}-1}$ vérifie $s^2 = s$. On dit alors que $s$ est idempotent.

\end{enumerate}

\item \begin{enumerate}

\item Ce résultat reste-t-il vrai si on retire l'hypothèse de finitude ?

\item Dans un groupe fini, a-t-on toujours un idempotent distinct du neutre ?

\end{enumerate}

\end{enumerate}





