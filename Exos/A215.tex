\exercice[Sous-groupes de $\mb{R}$]

\begin{enumerate}

\item Donner un exemple de sous-groupe de $(\mb{R}, +, 0)$ dense dans $\mb{R}$ et distinct de $\mb{Q}$.

\item Soit $G$ un sous-groupe non trivial de $(\mb{R}, +, 0)$ et $a = \inf G \cap \mb{R}^*_+$. On suppose dans cette question que $a > 0$.

\begin{enumerate}

\item Si $a \in G$, montrer que $G = a \mb{Z}$.

\emph{On montrera d'abord $a\mb{Z} \subseteq G$, puis dans un second temps $G \subseteq a \mb{Z}$ par l'absurde.}

\item On va montrer que $a \in \mb{G}$ nécessairement. Supposons le contraire. Montrer qu'il existe $x \in \mb{G}$ tel que $a< x \le 2a$. Conclure à une contradiction.

\end{enumerate}

\item On suppose maintenant que $a = 0$. Montrer que $G$ est dense dans $\mb{R}$.

\emph{Indication : faire des petits pas.}

\item Montrer que $ \exp : (\mb{R}, +, 0) \rightarrow (]0, + \infty[, \times, 1)$ est un isomorphisme de groupes.

Quelle est l'image de $a \mb{Z}$ ?

\end{enumerate}

