\documentclass[11pt]{article}
\usepackage[french]{babel}
\usepackage[T1]{fontenc}
\usepackage[utf8]{inputenc}
\usepackage{amsfonts, amsmath, amssymb,dsfont, mathrsfs}
\usepackage[a4paper,twoside]{geometry}
\geometry{left=2.5cm, right=2.5cm,top=1.5cm ,bottom=4cm}
\usepackage{fancyhdr}
\usepackage{graphicx}
\usepackage{amsthm}
\usepackage{multicol}
\usepackage{complexity}
\usepackage{hyperref}
%\usepackage{enumitem}
\usepackage{paralist}
\usepackage{epigraph}
\usepackage{tikz}
\usetikzlibrary{arrows,positioning,automata,shadows}
\usepackage{tikz-cd}
\usepackage{palatino}
\usepackage{pdfpages}
\usepackage{epigraph}
\usepackage{multicol}
\usepackage{xcolor}
\usepackage{pythontex}


\newcommand{\es}{\vspace{1\baselineskip}}
\newcommand{\ds}{\vspace{0.4\baselineskip}}
\newcommand{\di}{\mathrm{d}}
\newcommand{\ex}{{\operatorname{e}}}
\newcommand{\im}{\text{i}}
\newcommand{\mb}[1]{\mathbb{#1}}
\newcommand{\mc}[1]{\mathcal{#1}}
\newcommand{\mf}[1]{\mathfrak{#1}}
\newcommand{\card}{\operatorname{Card}}
\newcommand{\trace}{\operatorname{tr}}

\newcounter{exo}
\setcounter{exo}{1}
\newcommand{\exercice}[1][\null]{\textbf{\\ Exercice \theexo} \emph{(#1)}. \addtocounter{exo}{1}}
\newcommand{\resetexo}{\setcounter{exo}{1}}

\newcommand{\indication}[1]{\begin{flushright}\textit{Indication: #1}\end{flushright}}



\newenvironment{exo}{
\begin{center}
\begin{minipage}{0.8\linewidth}
}{
\end{minipage}
\end{center}
}

\newcommand{\writeExo}[1]{
    \begin{exo} 
        \py{writeExercice ("#1")}
    \end{exo} 
}

\newcommand{\qcours}[2]{\textbf{#1} & #2 \\}

\setlength{\headheight}{35pt}
\chead{}
\rfoot{Lopez Aliaume}
\pagestyle{fancy}
\lhead{MPSI -- Semaine 16}
\rhead{29 Janvier -- 2 Février 2018}


\begin{document}

    \begin{pycode}
pytex.add_dependencies ("query.py")
from query import *
import random
    \end{pycode}


\begin{center}
\rule{15em}{2pt}


\vspace{1em}
\textbf{\huge{Analyse Asymptotique \\ Matrices}}
\vspace{0.5em}


\rule{15em}{2pt}
\end{center}



    \begin{center}
        \rule[0.23em]{1.5em}{1pt} 
        \textbf{Question de cours}
        \rule[0.23em]{24em}{1pt}
    \end{center}
    

\begin{center}
\begin{minipage}{0.8\linewidth}
    \begin{pycode}
print ("\\begin{tabular}{rl}")
cours  = fetchCours ("semaine17")
eleves = cours["MARDI"]
quest  = [ random.choice (cours["QUESTIONS"]) for e in eleves ] 
for e,c in zip (eleves,quest):
    print ("\\qcours{" + e + "}{" + c +  "}")
print ("\\end{tabular}")
    \end{pycode}
\end{minipage}
\end{center}

    \begin{center}
        \rule{35em}{1pt}
    \end{center}

    \begin{pycode}
request   = {
    "tags" : { 
        "EXACT" : "Quantificateurs"
    },
    "difficultée" : {
        "EXACT" : "*",
        "EXACT" : "**"
    }
}
reqrun    = fetchExercices (request)
for exoID in reqrun:
    print ("\\textbf{" + exoID + "}")
    print (fetchExercice (exoID))
    \end{pycode}

\end{document}
