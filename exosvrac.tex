\documentclass{article}
\usepackage[french]{babel}
\usepackage[T1]{fontenc}
\usepackage[utf8]{inputenc}
\usepackage{amsfonts, amsmath, amssymb,dsfont, mathrsfs}
\usepackage [a4paper]{geometry}
\geometry{left=3.5cm, right=3.5cm,top=3.5cm ,bottom=4cm}
\usepackage{fancyhdr}
\usepackage{graphicx}
\usepackage{amsthm}
\usepackage{multicol}
\usepackage{complexity}
\usepackage{hyperref}
%\usepackage{enumitem}
\usepackage{paralist}
\usepackage{epigraph}
\usepackage{tikz}
\usetikzlibrary{arrows,positioning,automata,shadows}
\usepackage{palatino}
\usepackage{pdfpages}
\usepackage{epigraph}
\usepackage{multicol}

%%%% Format

\setlength{\headheight}{35pt}
\lhead{Colles\\MPSI}
\rhead{}
\chead{}
\rfoot{}
\pagestyle{fancy}

\newcommand{\fonction}[5]{\raisebox{-0.5\baselineskip}{$\begin{array}{lccc}
    #1: & #2 & \longrightarrow & #3 \\
        & #4 & \longmapsto & #5 \end{array}$}}
        
\newcommand{\es}{\vspace{1\baselineskip}}
\newcommand{\ds}{\vspace{0.4\baselineskip}}
\newcommand{\di}{\mathrm{d}}
\newcommand{\ex}{{\operatorname{e}}}
\newcommand{\im}{\text{i}}
\newcommand{\mb}[1]{\mathbb{#1}}
\newcommand{\mc}[1]{\mathcal{#1}}
\newcommand{\mf}[1]{\mathfrak{#1}}

% Exercices

\newcounter{exo}
\setcounter{exo}{1}

\newcommand{\exercice}[1][\null]{\textbf{\\ Exercice \thesection.\theexo. #1} \addtocounter{exo}{1}}
\newcommand{\cours}{\textbf{\ds \\ \large Questions de cours}}

%%%% Corps du fichier


\begin{document}



\title{Exercices détaillés de MSPI}

\author{Aliaume Lopez \& Gaëtan Douéneau}

\maketitle

\tableofcontents

\pagebreak


\section{Logique et algèbre}

\subsection{Logique et raisonnement}

\subsubsection{Quantificateurs}

\exercice 

Soit $f : \mb{R} \rightarrow \mb{R}$ une fonction. A-t-on toujours $\forall x \in \mb{R}, \exists y \in \mb{R}, f(x) = y$ ? Donner la négation de cette formule. Que dire des fonctions qui vérifient $\exists y \in \mb{R}, \forall x \in \mb{R},  f(x) = y$ ?

\exercice[Sous-structures]

On se donne une proposition $P(x)$ dépendante d'un réel $x$.

\begin{enumerate}

\item Soit $X \subseteq \mb{R}$ telle que $\forall x \in H, P(x)$. Soit $I \subseteq H$, montrer que $\forall x \in I, P(x)$.

\item Soit $J \subseteq \mb{R}$ telle que $\exists x \in H, P(x)$. Soit $J \subseteq K \subseteq \mb{R}$, montrer que $\exists x \in J, P(x)$.
\end{enumerate}

\exercice[Logique élémentaire]

\begin{enumerate}
    \item Prouver la formule suivante~: $\forall x \in \mathbb{R}, \exists y \in
        \mathbb{R}, x + y = 0$

    \item Donner la négation de cette formule. Si cette négation est vraie
        donner une preuve et si non donner un contre exemple explicite.

    \item Que dire de la formule suivante~: $\forall x \in \mathbb{R}, \exists y \in
        \mathbb{R}, xy = 1$ ?
\end{enumerate}

\exercice[Contre exemple à la distributivité des quantificateurs]

Trouver deux propositions $P(x)$ et $Q(x)$ telles que la propriété 
$(1)$ soit vraie, mais que la propriété $(2)$ soit fausse.

\begin{enumerate}[(1)]
    \item $\forall x \in \mathbb{R}, (P (x) \vee Q(x))$
    \item $\left(\forall x \in \mathbb{R}, P(x)\right) 
        \vee 
        \left(\forall x \in \mathbb{R}, Q(x)\right)$
\end{enumerate}

\exercice[Borne inférieure]

Montrer que $\forall x \in \mb{R}, \forall y \in \mb{R}, \left[ (\forall \varepsilon > 0, x \le y + \varepsilon) \Rightarrow (x \le y) \right]$.

\exercice[Croissance des fonctions]

On dit que $f : I \to \mb{R}$ est croissante 
quand elle vérifie la propriété suivante~:

\begin{equation}
    \forall (x,y) \in I^2, x \leq y \implies f(x) \leq f(y)
\end{equation}

On dit qu'une fonction $f : I \to \mb{R}$ est \emph{strictement}
croissante quand elle vérifie la propriété de croissance avec 
des inégalités strictes.

\begin{enumerate}
    \item Que dire si l'on remplace l'implication par une équivalence 
        dans la définition de croissance ?
    \item Que dire si l'on remplace l'implication par une équivalence 
        dans la définition de la croissance stricte ?
\end{enumerate}


\subsubsection{Démonstrations basiques}

\exercice Donner les fonctions $f : \mb{N} \rightarrow \mb{N}$ telles que pour tous $a,b$ dans $\mb{N}$, $f(a+b) = f(a)+f(b)$.



\exercice  Montrer que toute fonction $f : \mb{R} \mapsto \mb{R}$ s'écrit comme somme d'une fonction paire et d'une fonction impaire. Prouver l'unicité de cette écriture.


\exercice

Soient $A$ et $B$ deux parties de $E$. Montrer que $A \cap B = A \cup B$ implique $A = B$.




\exercice

Trouver toutes les fonctions $f : \mb{Q}_{+}^* \rightarrow \mb{Q}_{+}^*$ vérifiant pour tous $a,b$, $f(ab) = f(a)f(b)$ et $f(a+b) = f(a)+f(b)$ .



\exercice

Soit $f : \mb{R} \rightarrow \mb{R}$ une fonction $T$-périodique pour tout $T \ge 0$. Montrer qu'elle est constante. (Une fonction $f$ est dite $T$-périodique si pour tout $x \in \mb{R}$, $f(x) = f(T+x)$).

\exercice

\begin{enumerate}

\item Montrer que $\sqrt{2} \not \in \mb{Q}$.

\item On veut montrer que $\exists x, y \not \in \mb{Q}, x^y \in \mb{Q}$.

\begin{enumerate}

\item On suppose que $\sqrt{2}^{\sqrt{2}} \not \in \mb{Q}$. Montrer la propriété demandée.

\item Conclure.

\end{enumerate}



\end{enumerate}


\exercice[Sur les inclusions d'ensembles]

On pose $P = \{ 2k ~|~ k \in \mathbb{N} \}$ et $I = \{ 2k + 1 ~|~ k \in
\mathbb{N} \}$.

\begin{enumerate}
    \item Montrer par récurrence sur $n \in \mathbb{N}$ la propriété suivante~:
        \begin{equation}
            \forall n \in \mathbb{N}, \exists k \in \mathbb{N}, (n = 2k) \vee 
            (n = 2k + 1)
        \end{equation}
    \item En déduire une inclusion entre $P \cup I$ et $\mathbb{N}$
    \item Montrer l'inclusion réciproque
    \item En déduire que $P \cup I = \mathbb{N}$
    \item Montrer que $P \cap I = \emptyset$
    \item On dit qu'un entier est pair quand il est dans $P$ et qu'un 
        entier est impair quand il est dans $I$. Qu'avons nous montré 
        dans cet exercice ?
\end{enumerate}


\subsection{Fonctions et ensembles}

\subsubsection{Formalisme}

\exercice[Pré-images de fonctions]

\begin{enumerate}
    \item Si $f : x \mapsto x^2$, calculer 
        $f^{-1}( [-1;4])$, $f^{-1}([-1;0])$
        et $f^{-1}([0;4])$

    \item Montrer que si $f : E \to F$ alors 
        si $A$ et $B$ sont des parties de $F$,
        $f^{-1}(A \cap B) = f^{-1} (A) \cap f^{-1}( B)$

    \item Montrer que si $f : E \to F$ alors 
        si $A$ et $B$ sont des parties de $F$,
        $f^{-1}(A \cup B) = f^{-1} (A) \cup f^{-1}( B)$

    \item Est-ce vrai des images directes en général ?

    \item Montrer que $f : E \to F$ est injective 
        si et seulement si $\forall A \in \mathcal{P}(E),
        A = f^{-1}(f(A))$

    \item Montrer que $f : E \to F$ est surjective 
        si et seulement si $\forall B \in \mathcal{P}(F),
        B = f( f^{-1} (B))$.
\end{enumerate}



\exercice 
Soit $f : E \rightarrow E$. Soit $A \subseteq E$, on définit par récurrence la suite d'ensembles $(A_n)_{n \ge 0}$ par $A_0 = A$ puis $A_n = f(A_{n-1})$ pour $n \ge 1$. Soit $\displaystyle B = \bigcup_{n \in \mb{N}} A_n$.

\begin{enumerate}

\item Montrer que $f(B) \subseteq B$.

\item Soit $C \subseteq E$ telle que $A \subseteq C$ et $f(C) \subseteq C$. Montrer que $B \subseteq C$.
\end{enumerate}


\exercice 

Montrer que $D = \{ (x,y) \in \mb{R}^2 ~|~ x^2 + y^2 = 1 \}$
ne peut pas s'écrire comme un produit cartésien $A \times B$
avec $A$ et $B$ des parties de $\mb{R}$.


\exercice[Fonctions indicatrices]

Soit $E$ un ensemble. Si $A \subseteq E$ on pose 
$\mathds{1}_A (x) = 0$ quand $x \not \in A$ et $\mathds{1}_A (x) = 1$ 
quand $x \in A$.

\begin{enumerate}
    \item Justifier que si $A$ est une partie de $E$
        $\mathds{1}_A$ est bien une fonction de $E$ dans $\{ 0, 1 \}$

    \item Étant donné une fonction $f$ de $E$ dans $\{ 0, 1 \}$
        construire un ensemble $A$ tel que $f = \mathds{1}_A$

    \item En déduire une bijection entre $\mathcal{P}(E)$
        et $E \to \{ 0, 1 \}$

    \item Justifier que si $f_1$ et $f_2$ sont des 
        fonctions de $E$ dans $\{0,1\}$ alors 
        les opérations suivantes définissent bien 
        une fonction de $E$ dans $\{ 0, 1\}$:


        \begin{multicols}{2}
            \begin{enumerate}
                \item $\max (f_1, f_2)$
                \item $\min (f_1, f_2)$
                \item $f_1 \times f_1$
                \item $1 - f_1$
                \item $f_1 + f_2 - f_1 \times f_2$
                \item $(f_1 - f_2)^2$
            \end{enumerate}
        \end{multicols}


    \item En utilisant la bijection, étudier 
        le sens de ces opérations dans $\mathcal{P}(E)$
\end{enumerate}

\exercice[Morphismes]

Soit $f : I \to \mb{R}$ avec $I \subseteq \mb{R}$.
On suppose que $f(x + y) = f (x) + f(y)$,
que $0 \in I$ et que $I$ est stable par addition.

\begin{enumerate}
    \item Montrer que $f(I)$ contient $0$
        et est stable par addition
    \item Si on suppose de plus que $I$ est 
        stable par soustraction, montrer 
        que $f(I)$ est stable par soustraction.
    \item On pose $J$ un sous ensemble de $\mb{R}$
        stable par addition, soustraction et qui 
        contient $0$, montrer que $f^{-1}(J)$
        vérifie encore ces propriétés 
\end{enumerate}




\subsubsection{Injection, surjection, bijection}

\exercice[Caractérisations des injections]

Montrer que pour $f : A \rightarrow B$, on a équivalence entre les trois assertions suivantes.

\begin{itemize}

\item $f$ est injective ;

\item pour tout $X,Y \subseteq A$, $f(X \cap Y) = f(X) \cap f(Y)$ ;

\item pour tout $X \subseteq A$, $f^{-1}(f(X)) = X$.

\end{itemize} 

\exercice[Épi/mono]

\begin{enumerate}
    \item Montrer que $f: E \to F$ est surjective 
        si et seulement si 
        pour toute paire $(g,h)$ de fonctions de $F$ vers $G$
        on a $g \circ f = h \circ f \implies g = h$.

    \item Montrer que $f : E \to F$ est injective 
        si et seulement si 
        pour toute parie $(g,h)$ de foncitons de $G$ vers $E$
        on a $f \circ g = f \circ h \implies g = h$.
\end{enumerate}



\exercice[Axiome du choix \dots{} ou pas]

\begin{enumerate}

\item Monter que $f : E \rightarrow F$ est injective, si et seulement si, il existe $g : F \rightarrow E$ telle que $f \circ g = \text{id}_E$.

\item Monter que $f : E \rightarrow F$ est surjective, si et seulement si, il existe $g : F \rightarrow E$ telle que $g \circ f = \text{id}_F$.

\end{enumerate}



\exercice[Bijectivité]

Montrer que $f : E \to F$ est bijective si et seulement 
si pour toute partie $A$ de $E$ on a $f(A^c) = f(A)^c$.

\exercice[Injectivité et monotonie]

\begin{enumerate}

\item Soit $f : \mb{R} \rightarrow \mb{R}$ strictement monotone. Montrer qu'elle est injective.

\item A-t-on la réciproque ? Et si on impose que $f$ soit continue ?

\end{enumerate}



\exercice[Une histoire de notations]

Soit $f : E \to E$ telle que $f(f(E)) = E$,
montrer que $f$ est surjective.

\exercice 

Déterminer une bijection de $[-1,1]$ dans $\mb{R}$.





\subsubsection{Cardinalité}

\exercice[Finitude]

Soit $E$ un ensemble, montrer que les deux propositions suivantes 
sont équivalentes~:

\begin{enumerate}[(i)]
    \item $E$ est fini
    \item Toute fonction $f : E \to E$ admet une partie stable non 
        triviale
\end{enumerate}

Une partie stable est un sous ensemble $F \subseteq E$ tel 
que $f(F) \subseteq F$. Une partie triviale est une partie $F$
de $E$ qui est égale à $\emptyset$ ou $E$.




\exercice[Ensembles dénombrables]

\begin{enumerate}

\item Expliciter une bijection $f$ entre $\mb{N}$ et $\mb{Z}$.

\item Soit $\fonction{g}{\mb{N}^2}{\mb{N}}{(i,j)}{ i + \sum_{k=0}^{i+j} k}$.

\begin{enumerate}

\item Calculer $g(i,j)$ pour les $i,j$ tels que $i+j \le 3$, puis représenter les valeurs obtenues sur le plan $\mb{N}^2$. Décrire par un dessin le comportement général de $g$.

\item Montrer que $g$ est surjective.

\emph{Pour trouver un antécédent de $n$, on pourra considérer $p := \min \left\{ p \in \mb{N}~|~ \sum_{k=0}^{p} k \le n\right\}$}.

\item Montrer que $g$ est injective.

\end{enumerate}

\item En remarquant que $\forall n \ge 1$, $\mb{N}^{n+1} = \mb{N}^n \times \mb{N}$, construire par récurrence sur $n > 0$ une bijection entre $\mb{N}$ et $\mb{N}^n$.

\end{enumerate}




\exercice[Argument diagonal]

\begin{enumerate}

\item Donner une injection $g$ de $\mb{N}$ dans $\{0,1\}^{\mb{N}}$ (suites à valeurs dans $\{0,1\}$).

\item On va montrer qu'il n'existe pas de fonction surjective de $\mb{N}$ dans $\{0,1\}^{\mb{N}}$.

Soit $f : \mb{N} \rightarrow \{0,1\}^{\mb{N}}$, on pose $(u_{n,i})_{i \in \mb{N}} := f(n)$ pour $n \in \mb{N}$.

\begin{enumerate}

\item Expliciter $(u_{n,i})_{i \in \mb{N}}$ pour $n \ge 0$, dans le cas où $f = g$ de la question précédente.

\item Soit la suite $(v_i)_{i \in \mb{N}}$ définie par $v_i := 1 - u_{i,i}$. Justifier que $(v_i)_{i \in \mb{N}} \in \{0,1\}^{\mb{N}}$. 

\item Montrer que $\forall n \in \mb{N}$, $ f(n)\neq (v_i )_{i \in \mb{N}}$.

\item Conclure quant à la surjectivité de $f$ ? $\mb{N}$ et $\{0,1\}^{\mb{N}}$ peuvent-ils être en bijection ?

\end{enumerate}

\item Existe-t-il une surjection de $\mb{N}$ dans $\mc{P}(\mb{N})$ ?

\end{enumerate}





\exercice[Théorème de Cantor-Bernstein]

Soient $A$ et $B$ deux ensembles, on suppose qu'il existe une injection $f : A \rightarrow B$ et une injection $g : B \rightarrow A$. L'objectif est de montrer qu'il existe une bijection $h : A \rightarrow B$.

\begin{enumerate}

\item 

\end{enumerate}






\subsubsection{Propriétés universelles}

\exercice[Propriété universelle du produit]

\begin{enumerate}

\item Si $A$ et $B$ sont deux ensembles finis, quel est le cardinal de $A \times B$ ?

\item Soient $A$ et $B$ deux ensembles non vides.

On pose $\fonction{\pi_A}{A\times B}{A}{(a,b)}{a}$ et $\fonction{\pi_B}{A\times B}{A}{(a,b)}{b}$.

Montrer que $\pi_A$ et $\pi_B$ sont surjectives. Quand a-t-on injectivité de $\pi_A$ ?

\item Soit $C$ un ensemble et deux fonctions $f_A : C \rightarrow A$ et $f_B : C \rightarrow B$. Montrer qu'il existe une unique fonction $h : C \rightarrow A \times B$ telle que $\pi_A \circ h = f_A$ et $\pi_B \circ h = f_B$. 

\emph{On pourra raisonner par analyse-synthèse.}

\item Faire un dessin de la construction. On représentera les fonctions $\pi_A, \pi_B, f_A, f_B$ et $h$ par des flèches entre les ensembles $A, B, A \times B$ et $C$.

\end{enumerate}



\exercice[Propriété universelle du coproduit]

 Si $A$ et $B$ sont deux ensembles, on définit l'union disjointe $A \uplus B := \{0\} \times A \cup \{1\} \times B $

\begin{enumerate}

\item Si $A$ et $B$ sont deux ensembles finis, quel est le cardinal de $A \uplus B$ ? Celui de $A \times B$ ?
\item Soient $A$ et $B$ deux ensembles.

On pose $\fonction{\mu_A}{A}{A\uplus B}{a}{(0,a)}$ et $\fonction{\pi_B}{B}{A\uplus B}{b}{(1,b)}$.

Montrer que $\mu_A$ et $\mu_B$ sont injectives. Quand a-t-on surjectivité de $\mu_A$ ?

\item Soit $C$ un ensemble, et deux fonctions $f_A : A \rightarrow C$ et $f_B : B \rightarrow C$. Montrer qu'il existe une unique fonction $h : A \uplus B \rightarrow C$ telle que $h \circ \mu_A  = f_A$ et $h \circ \mu_B  = f_B$. 

\emph{On pourra raisonner par analyse-synthèse.}

\item Faire un dessin de la construction. On représentera les fonctions $\mu_A, \mu_B, f_A, f_B$ et $h$ par des flèches entre les ensembles $A, B, A \times B$ et $C$.

\end{enumerate}


\exercice[Propriété universelle du quotient]

Soit $A$ un ensemble et $\sim$ une relation d'équivalence sur $A$.

\begin{enumerate}

\item Montrer que les classes d'équivalence de $\sim$ forment une partition $(A_i)_{i \in I}$ de $A$. Dans la suite, on note $\overline{A} = \{A_i~|~i \in I\}$ l'ensemble des classes.

\item Soit $\fonction{\pi}{A}{\overline{A}}{x}{A_i \text{ tel que } x \in A_i}$.

\begin{enumerate}

\item Représenter (patates) $\overline{A}$ et $\pi$ dans le cas où $A = \{0,1,2\}$ avec $0 \sim 2$ et $1 \not \sim 2$.

\item Dans le cas général, pourquoi $\pi$ est-elle bien définie ?

\item Montrer que $\pi$ est surjective.

\end{enumerate}

\item Soit $B$ un ensemble et $f : A\rightarrow B$ telle que $\forall x,y \in A$, $x \sim y \Rightarrow f(x) = f(y)$.

Montrer qu'il existe une unique fonction $\overline{f} : \overline{A} \rightarrow B$ telle que $f = \overline{f} \circ \pi$.

\emph{On pourra raisonner par analyse-synthèse.}

\item Faire un dessin de la construction. On représentera les fonctions $\pi, f$ et $ \overline{f}$ par des flèches entre les ensembles $A, \overline{A}$ et $B$.
\end{enumerate}



\exercice

Etudier les variations de la fonction $\fonction{f}{\mc{P}(E)}{\mc{P}(E)}{X}{E \smallsetminus X}$






\subsection{Relations}

\subsubsection{Formalisme}

\exercice[Clôtures d'une relation]
 
 Soit $E$ un ensemble et $R \subseteq E^2$ une relation binaire. On note $\mc{R}(R)$ la plus petite relation réflexive qui contient $R$, appelée clôture réflexive de $R$. De même on note $\mc{S}(R)$ la clôture symétrique de $R$, et $\mc{T}(R)$ sa clôture transitive.\footnote{Indication générale : pour montrer que $X$ est le plus petit \emph{truc} vérifiant $P$, on procède en 2 étapes. Montrer que $X$ est un \emph{truc} vérifiant $P$. Puis montrer que tout \emph{truc} vérifiant $P$ contient $X$.}



\begin{enumerate}

\item Montrer que $\mc{R}(R)  = R \cup \{ (x,x) ~|~x\in E\}$.

\item Montrer que $\mc{S}(R)  = R \cup \{ (y,x) ~|~(x,y)\in R\}$. A-t-on $ \mc{R}(\mc{S}(R)) = \mc{S}(\mc{R}(R))$ ?

\item Soit $\fonction{f}{\mc{P}(E^2)}{\mc{P}(E^2)}{S}{S \cup \{(x,z)~|~\exists y, (x,y)\in S \text{ et }(y,z) \in S\}}$

Montrer que $\mc{T}(R) = \cup_{n \ge 0} f^n(R)$ où $f^n = f \circ f \circ \cdots \circ f$ ($n$ fois) et $f^0 = \text{id}$.

\emph{On prouvera par récurrence sur $n$ que $f^n(S) \subseteq R'$ pour toute $R'$ transitive contenant $R$.}

\item Dans cette question on suppose $R$ symétrique et réflexive. Soit $\mc{E}(R)$ la plus petite relation d'équivalence contenant $R$. Montrer $\mc{T}(R) = \mc{E}(R)$.

\emph{On utilisera la question précédente pour montrer que $\mc{T}(R)$ est une relation d'équivalence.}

\item On ne fait plus d'hypothèses sur $R$. Montrer que $\mc{E}(R) = \mc{T} (\mc{S}(\mc{R}(R)))$.

\item Application. Soit $E = \{1 \dots 6\}$ et $R = \{(1,2),(2,3),(5,6)\}$. Expliciter $\mc{E}(R)$ puis ses classes d'équivalence.


\end{enumerate}




\exercice[Vision graphique]

Soit $f : E \to F$ une \emph{fonction}.

\begin{itemize}
    \item Dans le cas où $E = F = \mb{R}$ 
        définir le graphe de $f$ noté $\Gamma_f$ 
        comme un sous ensemble de $\mb{R}^2$.

    \item Généraliser cette définition 
        au cas où $E$ et $F$ sont quelconques 

    \item Montrer que si $f$ est une fonction 
        $\Gamma_f$ est une relation de $E \times F$ 
        qui vérifie
        \begin{enumerate}[(i)]
            \item $\forall x \in E, \exists y \in F, x \Gamma_f y$
            \item $\forall x \in E, \forall (y_1,y_2) \in F^2, 
                x \Gamma_f y_1 \wedge x \Gamma_f y_2 \implies 
                y_1 = y_2$
        \end{enumerate}

    \item Si $R$ est une relation de $E \times F$
        on pose $R^{op}$ une relation de $F \times E$
        définie par $\{ (y,x) \in E\times F ~|~ x R y \}$

        Montrer que si $f$ est bijective, $\Gamma_f^{op} = \Gamma_{f^{-1}}$.

    \item En général si $f$ est une fonction, quelle 
        à quelle propriété de $\Gamma_f^{op}$ correspond la 
        \emph{surjectivité} ? \emph{L'injectivité} ? On 
        se servira des propriétés $(i)$ et $(ii)$ définies 
        sur les relations.
\end{itemize}


\exercice[Résolution d'équations]

Soit $f : E \to F$ une fonction.

\begin{itemize}
    \item Soit $y \in F$, quel est l'ensemble 
        qui résout $f(x) = y$ ?

    \item Soit $g : F \to G$ et $y \in G$, quel
        est l'ensemble qui résout $g(f(x)) = y$ ?

    \item Montrer que $(g \circ f)^{-1}(A) = f^{-1} (g^{-1} (A))$

    \item Résoudre l'équation $\exp ((x + 1)^2) \in [1; 100]$

\end{itemize}





\subsubsection{Relations d'ordre}

\exercice[Symétrie des bornes]

Soit $(E, \le)$ un ensemble ordonné. On suppose que toute partie non vide et majorée de $E$ admet une borne supérieure.

\begin{enumerate}

\item Soit $A \neq \varnothing$ une partie minorée de $E$. Montrer que $B := \{x~|~ \forall y \in A, x \le y\}$  admet une borne supérieure.

\item On va montrer que $A$ admet une borne inférieure. 
\begin{enumerate}

\item Montrer que $\sup(B)$ est un minorant de $A$. \emph{On remarquera que $A$ est contenu dans l'ensemble des majorants de $B$}.

\item Montrer que $\sup(B)$ est le plus petit d'entre eux et conclure.

\end{enumerate}

\item On suppose maintenant que $E$ possède un plus grand et un plus petit élément. Montrer que toute partie de $E$ admet des bornes supérieure et inférieure.

\end{enumerate}



\exercice[Prolongement des applications]

Soient $A,B$ deux ensembles et $\mc{E} = \{(F,f)~|~F \subseteq A \text{ et } f : F \rightarrow B\}$.

\begin{enumerate}

\item Soit $\preccurlyeq$ défini par $(F,f) \preccurlyeq (G,g)$ ssi $F \subseteq G$ et $\forall x \in F, f(x) = g(x)$ ($g$ est un prolongement de $f$). Montrer que $\preccurlyeq$ une relation d'ordre sur $\mc{E}$.

\item A quelles conditions sur $A,B$ l'ordre est-il total ?

\item Montrer que $\mc{E}$ possède un plus petit élément.

\item Soit $\mc{F}\subseteq \mc{E}$ un sous-ensemble totalement ordonné de $\mc{E}$. Montrer que $\mc{F}$ est majoré.

\emph{Remarque : on dit que $\mc{E}$ est un ensemble inductif.}

\end{enumerate}





\exercice[Induction]

\begin{itemize}
    \item Définition d'un ordre partiel, d'un ordre total
    \item Soit $E$ un ensemble totalement ordonné, 
        est-ce que toute partie de $E$ admet un minimum ?
        (\emph{Indication: $\mb{Q}$}) 
    \item Montrer que toute partie non vide de $\mb{N}$ 
        admet un plus petit élément (\emph{on admet 
        le raisonnement par récurrence valide})
    \item Montrer le principe de récurrence sur $\mb{N}$
        en utilisant le fait qu'une partie non vide 
        admet un plus petit élément
\end{itemize}



\section{Calcul algébriques}

\subsection{Sommes et produits de nombres réels}

\subsubsection{Calculs classiques}


\exercice[Autour de la somme des $k$]

Notons $S_n = \sum_{k=1}^n k $ pour $n \ge 1$.

\begin{enumerate}

\item Donner la formule du cours (sans somme) pour $S_n$, puis la prouver par récurrence.

\item En calculant $2 S_n$, re-prouver ce résultat sans récurrence.

\item Re-re-prouver ce résultat en calculant $S_n = \sum_{k=1}^n (k+1)^2 -  \sum_{k=1}^n k^2  $ de deux manières différentes.
\end{enumerate}


\exercice

\begin{enumerate}

\item Rappeler la formule reliant $\sin(2x)$ avec $\cos(x)$ et $\sin(x)$.

\item Soit $x \in ]0, \pi [$, simplifier $P =\displaystyle \prod_{k=0}^n \cos(2^k x)$ (on commencera par calculer $ \sin (x) \times P$).

\end{enumerate}

\exercice[Suite croissante]

Montrer que la suite $S_n$ définie ci-dessous est strictement croissante.

\begin{equation*}
    u_n = \sum_{k = 1}^n \frac{1}{n + k}
\end{equation*}

\exercice[Petites variations]

\begin{enumerate}
    \item Calculer $S_n = \sum_{k =1}^{n} k 2^{-k}$ de deux manières 
        différentes
    \item En déduire une expression de  $T_n = \sum_{k=1}^{n} k 2^{-k+1}$
\end{enumerate}


\exercice  Calculer $\displaystyle \sum_{1 \le i, j \le n} ij$.

\exercice

Montrer que $\displaystyle \sum_{i=0}^n i x^i = \frac{n x^{n+2} - (n+1)x^{n+1} +x}{(x-1)^2}$

\exercice

Montrer que $\displaystyle \prod_{i=0}(1-a_i) \ge 1- \sum_{i=0}^n a_i$ si $0<a_i<1$.

\exercice[Série harmonique]

Soit $ H_n := \sum_{k =1}^n 1/k$ (série harmonique).

\begin{enumerate}

\item Calculer $H_n$ pour $1\le n \le 5$.

\item Montrer que $H_n$ n'est jamais entier. On conjecturera, grâce à la question précédente, une certaine propriété à prouver par récurrence forte.

\end{enumerate}



\subsubsection{Coefficients binomiaux}

\exercice

Soit $\displaystyle S_1 = \sum_{\substack{k=0 \\ k \text{ pair}}}^n {n \choose k}$ et $\displaystyle S_2 = \sum_{\substack{k=0 \\ k \text{ impair}}}^n {n \choose k}$. Calculer $S_1$ et $S_2$.

\exercice[Formule d'inversion de Pascal]

\begin{enumerate}

\item Soit $n > 0$, calculer $\displaystyle \sum_{k=0}^n (-1)^k {k \choose n}$. Que dire si $n = 0$ ?

\item Soient $l \le k \le n$ des entiers, montrer que $\displaystyle {n \choose k} {k \choose l} = {n \choose l} {n-l \choose k-l} $.

\item Soit $(x_n)_{n \ge 0}$ une suite réelle, on pose $(y_k)_{k \ge 0}$ la suite définie par $\displaystyle y_k = \sum_{l=0}^k {k \choose l} x_l$.

Montrer que $\displaystyle x_n = \sum_{k=0}^n (-1)^{n-k} {n \choose k} y_k$.

\end{enumerate}


\exercice

Montrer que pour tout $n \ge 0$ on a $\displaystyle \sum_{k=1}^n \frac{(-1)^{k+1}}{k} {n \choose k} = \sum_{k=0}^n \frac{1}{k}$.



\subsubsection{Identités diverses}






\exercice[Une version simple de l'inégalité de Jensen]

Soit $f: \mb{R} \mapsto \mb{R}$ une fonction vérifiant pour tous réels $x_1, x_2$ et $t \in [0,1]$
$$f(t x_1 +(1-t) x_2) \le t f(x_1) +(1-t) f(x_2)$$  (une telle fonction est dite convexe).
Montrer que si les $\lambda_1 \dots \lambda_n$ sont des réels positifs ou nuls tels que $\sum_{i=1}^n \lambda_i = 1$, alors pour tous réels $x_1 \dots x_n$ on a :
$$f\left(\sum_{i=1}^n \lambda_i x_i\right) \le \sum_{i=1}^n \lambda_i f(x_i).$$


\exercice[Inégalité de Bernoulli]

\begin{enumerate}

\item Montrer par récurrence que $(1+x)^n \ge 1+nx$ pour $n \ge 1$ et $x \ge -1$.

\item Prouver ce résultat sans récurrence (pour $x \ge 0$).

\end{enumerate}



\exercice[Encadrement de la factorielle (Gauss)]

On va donner un encadrement de $n !$.

\begin{enumerate}

\item Montrer que $\displaystyle n! = \prod_{\substack{i,j \ge 1\\i+j = n+1}}  \sqrt{ij}$

\item Montrer que pour tous $i,j \ge 1$ on a $i+j-1 \le ij \le (\frac{i+j}{2})^2$.

\item Conclure que $n^{\frac{n}{2}} \le n ! \le (\frac{n+1}{2})^n$.

\end{enumerate}



\exercice[Binet, Lagrange, Cauchy-Schwarz]

Soient $x_1 \dots x_n, y_1 \dots y_n, a_1 \dots a_n, b_1 \dots, b_n \in \mb{R}$

\begin{enumerate}

\item Développer $\displaystyle \left(\sum_{i=1}^n a_i \right)\left(\sum_{i=1}^n b_i \right)$.

\item Soient $x_1 \dots x_n, y_1 \dots y_n, a_1 \dots a_n, b_1 \dots, b_n \in \mb{R}$. Montrer la formule de Binet :
$$ \left(\sum_{i=1}^n a_i x_i\right)\left(\sum_{i=1}^n b_i y_i\right) = \displaystyle \left(\sum_{i=1}^n a_i y_i\right)\left(\sum_{i=1}^n b_i x_i\right) + \sum_{1 \le i < j \le n} (a_i b_j - a_j b_i)(x_i y_j - x_j y_i).$$

\item En déduire l'identité de Lagrange :

$$\displaystyle \left(\sum_{k=1}^n a_k^2\right)\left(\sum_{k=1}^n b_k^2\right) - \left(\sum_{k=1}^n a_k b_k\right)^2 = \sum_{1 \le i < j \le n} (a_i b_j - a_j b_i)^2$$


\item Etablir l'inégalité de Cauchy-Schwarz :
$$\displaystyle \left(\sum_{k=1}^n a_k b_k\right)^2 \le \left(\sum_{k=1}^n a_k^2\right)\left(\sum_{k=1}^n b_k^2\right)$$

\item En déduire que $ \displaystyle \left(\sum_{k=1}^n c_k\right)\left(\sum_{k=1}^n \frac{1}{c_k}\right) \ge n^2$ si $c_1 \dots c_k$ sont des réels strictement positifs. Interpréter ce résultat en termes d'aires.

\end{enumerate}


\exercice[Interpolation de Lagrange revisitée]

Soient $x_1 \dots x_n \in \mb{R}$ tous distincts. Notons $\displaystyle P_i =  \prod_{\substack{1 \le j \le n \\ j \neq i}} (x_i - x_j)$.

Montrer que $\displaystyle \sum_{i=1}^n \frac{(x_i)^r}{P_j} = 0$ si $0 \le r < n-1$,  $=1$ si $r = n-1$, $=\displaystyle \sum_{i=1}^n x_i$ si $r = n$.









\subsection{Nombres complexes et trigonométrie}



\subsubsection{Module, argument, calculs}

\exercice

Soit $Z \in \mb{C}$, résoudre dans $\mb{C}$ l'équation $\text{e}^z  = Z$ d'inconnue $Z$.


\exercice

Trouver toutes les fonctions $f : \mb{C} \rightarrow \mb{C}$ telles que $\forall z~\in \mb{C} = f(z) + \text{i} f(\overline{z}) = 2i$.

\exercice

Soit l'application $f : z \mapsto \frac{1}{1-z}$.

\begin{enumerate}

\item Exprimer $f(\text{e}^{\text{i}\theta})$ sous la forme $a + \text{i} b$.

\item Montrer que $f$ est une bijection de $\mb{U}$ vers une droite que l'on précisera.

\end{enumerate}




\exercice

Soient $z$ et $z'$ deux nombres complexes de module $1$. Montrer que $\displaystyle \frac{z+z'}{1+zz'}$ est un réel (quand il est défini).



\exercice

Trouver en fonction de $p,q \in \mb{Z}$, les $z \in \mb{C}$ tels que $z^p = (z+1)^q = 1$.

\exercice

Quels sont les complexes qui peuvent s'écrire sous la forme $\frac{1+ \im r}{1- \im r}$ pour $r \in \mb{R}$ ?

\exercice

Simplifier $\displaystyle P = \prod_{k=0}^n(z^k + \overline{z}^{k})$ en fonction du module et de l'argument de $z$.




\exercice

On rappelle que l'exponentielle complexe est définie par $$\text{e}^{a+\text{i}b} := \text{e}^{a} \text{e}^{\text{i}b} = \text{e}^{a} (\cos(b) + \text{i} \sin(b)).$$

\begin{enumerate}


\item Soit $f: \mb{C} \rightarrow \mb{C}, z \mapsto \frac{\text{e}^{\text{i}z}+\text{e}^{-\text{i}z}}{2}$ et $g: \mb{C} \rightarrow \mb{C}, z \mapsto \frac{\text{e}^{\text{i}z}-\text{e}^{-\text{i}z}}{2\text{i}}$ .

Que valent $f(x)$ et $g(x)$ quand $x \in \mb{R}$ ? De quelles fonctions réelles $f$ et $g$ sont-elles les "extensions" aux nombres complexes ?

\item Montrer que $f(z)^2 + g(z)^2 = 1$. Cela est-il cohérent avec la question précédente ?

\item Prouver la formule de Moivre complexe : $\text{e}^{\text{i}z} = f(z) + \text{i} g(z)$.

\end{enumerate}




\exercice


On rappelle que l'exponentielle complexe est définie par $$\text{e}^{a+\text{i}b} := \text{e}^{a} \text{e}^{\text{i}b} = \text{e}^{a} (\cos(b) + \text{i} \sin(b)).$$
\begin{enumerate}


\item Soit $f: \mb{C} \rightarrow \mb{C}, z \mapsto \frac{\text{e}^{z}+\text{e}^{-z}}{2}$ et $g: \mb{C} \rightarrow \mb{C}, z \mapsto \frac{\text{e}^{z}-\text{e}^{-z}}{2}$ .

Que dire $f(x)$ et $g(x)$ quand $x \in \mb{R}$ ?

\item Montrer que $f(z)^2 - g(z)^2 = 1$.

\item Calculer $f(\text{i} x)$ et  $\frac{g(\text{i} x}{\text{i}}$ quand $x \in \mb{R}$.

\end{enumerate}






\exercice Calculer les sommes suivantes pour $n \ge 0$ et $\alpha, \beta \in \mb{R}$.

\begin{enumerate}

\item $\displaystyle \sum_{k=0}^n \sin(\alpha + k\beta)$ et $\displaystyle \sum_{k=0}^n \cos(\alpha + k\beta)$. 

\item $\displaystyle \sum_{k=0}^n {k \choose n} \sin(\alpha + k\beta)$ et $\displaystyle \sum_{k=0}^n {k \choose n} \cos(\alpha + k\beta)$.

\item $\displaystyle \sum_{k=0}^n \text{sh}(\alpha + k\beta)$ et $\displaystyle \sum_{k=0}^n \text{ch}(\alpha + k\beta)$. 

\item $\displaystyle \sum_{k=0}^n {k \choose n} \text{sh}(\alpha + k\beta)$ et $\displaystyle \sum_{k=0}^n {k \choose n} \text{ch}(\alpha + k\beta)$.

\end{enumerate}





\subsubsection{Racines de l'unité}



\exercice 
On note $\mathbb{U}_n$ l'ensemble des racines n-èmes de l'unité dans
$\mathbb{C}$.

Montrer que $\mathbb{U}_n = \{ e^{2ik\pi / n} ~|~ 0 \leq k \leq n - 1 \}$.



\exercice

On note $\mathbb{U}_n$ l'ensemble des racines n-èmes de l'unité dans
$\mathbb{C}$.
\begin{enumerate}
    \item Calculer $\sum_{ \omega \in \mathbb{U}_n} \omega$
    \item Calculer $\sum_{ \omega \in \mathbb{U}_n} | 1 - \omega |$
\end{enumerate}


\exercice

Soit $n \ge 1$. Trouver les $z \in \mb{C}$ tels que $(z-1)^n = (z+1)^n$.








\exercice[Autour des racines 3-èmes]

\begin{enumerate}

\item On note $\text{j} = \text{e}^{\frac{2 \text{i} \pi}{3}}$. Que valent $j^3$ et $j^4$ ? Les représenter sur le cercle unité.

\item Quelles sont les solutions dans $\mb{C}$ l'équation $1 + z + z^2 = 0$.

\item Soit $f : \mb{C}^3 \rightarrow \mb{C}$ définie par $f(z_1, z_2, z_3) = \left[z_1 + \text{j} z_2 + \text{j}^2 z_3\right]^3$. Montrer que pour tous $z_1, z_2, z_3 \in \mb{C}$ on a $f(z_1, z_2, z_3) = f(z_2, z_3, z_1) = f(z_3, z_1, z_2)$.

\end{enumerate}

\exercice[Un peu plus de $j$]

On rappelle que $j = e^{2i\pi /3}$.
On considère les trois somme suivantes~:

\begin{equation*}
    A_n = \sum_{k = 0 \wedge k \equiv 0 [3]}^{n} { n \choose k} 
\end{equation*}

\begin{equation*}
    B_n = \sum_{k = 0 \wedge k \equiv 1 [3]}^{n} { n \choose k} 
\end{equation*}

\begin{equation*}
    C_n = \sum_{k = 0 \wedge k \equiv 2 [3]}^{n} { n \choose k} 
\end{equation*}

\begin{enumerate}
    \item Calculer $S_n = \sum_{k = 0}^n { n \choose k } j^k$
    \item Montrer que $S_n = A_n + j B_n + j^2 C_n$
    \item En déduire que $\overline{S_n} = A_n + j^2 B_n + jC_n$
    \item Calculer $A_n + B_n + C_n$
    \item En dédure une expression de $A_n$, $B_n$ et $C_n$
\end{enumerate}










\exercice[Racines primitives de l'unité]

\begin{enumerate}

\item Soit $z \in \mb{U}_n$, montrer que $\{z^k~|~k \ge 0\} \subseteq \mb{U}_n$.

\item On dit que $z \in \mb{U}_n$ est une racine \emph{primitive} dans $\mb{U}
_n$ si  $\{z^k~|~k \ge 0\} = \mb{U}_n$. Quelles sont les racines primitives dans  $\mb{U}_1$ ? $\mb{U}_2$? $\mb{U}_3$? $\mb{U}_4$ ?

\item Montrer que si $z \in \mb{U}_n$ alors $\{z^k~|~k \ge 0\} = \{z^k~|~0 \le k < n\}$.

\item  Soit $z \in \mb{U}_n$ telle qu'il existe $ 0 < k < n$ avec $z^k = 1$. Montrer que $z$ n'est pas primitive dans $\mb{U}_n$.
\emph{Indication : compter le nombre d'éléments de $\{z^k~|~0 \le k < n\}$.}

\item Supposons maintenant que $z \in \mb{U}_n$ n'est pas primitive dans $\mb{U}_n$.  \begin{enumerate} 

\item Montrer qu'il existe $0 \le i\neq j < n$ tels que $z^i = z^j$.
\item En déduire qu'il existe $0 < k < n$ tel que $z^k = 1$.

\end{enumerate}

\item Conclure que $z \in \mb{U}_n$ est primitive dans $\mb{U}_n$ si, et seulement si, $\forall k < n, z \not \in \mb{U}_k$.

\end{enumerate}











\subsubsection{Applications}

\exercice[Fonctions symétriques de 2 variables]

Soient $z, z' \in \mb{C}$ on note $S = z + z'$ et $P = z z'$.

\begin{enumerate}

\item Exprimer $z^2 + z'^2$ en fonction de $P$ et $S$. Que peut-on en déduire si $P$ et $S$ sont réels ?

\item L'écriture précédente est de la forme $\sum_{i=1}^n \lambda_i P^{j_i} S^{j'_i}$ avec $\lambda_i \in \mb{R}$ et $j_i, j'_i \in \mb{N}$. Montrer que $z^2 + z'$ ne peut pas s'écrire de cette manière. On pourra prendre $z = \text{i}$ et $z' = - \text{i}$.

\item Exprimer $z^2 z' + z'^2z $ en fonction de $P$ et $S$. En déduire un expression de $z^3 + z'^3$.

\item Que peut-on dire pour $z^n + z'^n$ ?

\end{enumerate}


\exercice[Homothéties dans le plan complexe]

On considère $f$ une fonction qui vérifie la propriété 
suivante~:

\begin{equation*}
    \forall (z,z') \in \mathbb{C}^2,
    \forall (\alpha, \beta) \in \mathbb{R}^2,
    f( \alpha z + \beta z') = \alpha f (z) + \beta f(z')
\end{equation*}

\begin{enumerate}
    \item Montrer que $f(0) = 0$
    \item Montrer que $f$ est entièrement déterminée 
        par $f(1)$ et $f(i)$

    \item On suppose désormais que $f$ vérifie de plus 
        la propriété suivante~: $\forall z \in \mathbb{C},
        \exists \lambda \in \mathbb{R}, f (z) = \lambda z$.
        Montrer que $f$ vérifie alors~: $\exists \lambda
        \in \mathbb{R}, \forall z \in \mathbb{C}, f(z) = \lambda z$.
\end{enumerate}


\exercice[Transformée de Fourier Rapide]

On note $\alpha_0, \dots, \alpha_{n-1}$ des nombres réels.
On pose $\omega = e^{2i\pi / n}$, et 
pour $0 \leq 1 \leq n - 1$~:

\begin{equation*}
    \beta_i = \sum_{k = 0}^{n-1} \alpha_k (\omega^i)^k
\end{equation*}

\begin{enumerate}
    \item 
        Calculer $\sum_{i = 0}^{n - 1} \beta_i (\omega^{-1})^i$
    \item 
        Calculer $\sum_{i = 0}^{n-1} \beta_i (\omega^{-k})^i$

    \item En déduire qu'il est possible de récupérer les coefficients 
        $\alpha_k$ à partir des coefficients $\beta_k$.
\end{enumerate}


\exercice[Tcheby mon amour]

\begin{enumerate}
    \item Montrer que pour $n \in \mb{N}^+$ 
        et $\theta \in \mb{R}$ on a $\cos (n\theta) + i \sin (n \theta) = (\cos
        (\theta) + i \sin (\theta))^n$.

    \item En déduire une expression de $\cos (n\theta)$ (resp. $\sin (n\theta)$)
        en fonction de $\cos (\theta)$ (resp. $\cos (\theta)$)

    \item Montrer que $\cos (n \theta)$ est un polynôme de degré $n$ en $\cos
        (\theta)$ que l'on note $T_n$

    \item Calculer les racines de $T_n$
    \item Calculer les maxima et minima $x \mapsto T_n(x)$ sur $[-1,1]$

\end{enumerate}




\exercice[Inégalité triangulaire généralisée]

\begin{enumerate}

\item Montrer que pour tout $n \ge 1$, pour tous $z_1 \dots z_n \in \mb{C}$, on a $\displaystyle \left|\sum_{k=1}^n z_k\right| \le \sum_{k=1}^n | z_k |$.

On pourra utiliser, sans la redémontrer, l'inégalité triangulaire du cours.

\item Rappeler dans quel cas $|z_1 + z_2| = |z_1| + |z_2|$.

\item Montrer que s'il existe $1 \le i,j \le n$, $|z_i + z_j| < |z_i| + |z_j|$, alors $ \left|\sum_{k=1}^n z_k\right| < \sum_{k=1}^n | z_k |$ 

\item Déduire de 2 et 3 à quelles conditions l'inégalité de la question 1 est une égalité. L'interpréter géométriquement.

\end{enumerate}


\exercice[Entiers de Gauss]

On considère l'ensemble $\mb{Z}[\text{i}] := \{a + \text{i}b~|~a, b \in \mb{Z}\}$.

\begin{enumerate}

\item Montrer que si $z, z' \in \mb{Z}[\text{i}]$, alors $zz' \in \mb{Z}[\text{i}]$ et $z+z' \in \mb{Z}[\text{i}]$.

\item On cherche les $z \in \mb{Z}[\text{i}] $ tels que $ \frac{1}{z} \in \mb{Z}[\text{i}]$. De tels éléments sont dits \emph{inversibles}.

\begin{enumerate}

\item Montrer que $\text{i}$ est inversible, mais pas de $1+ \text{i}$.

\item Que dire de $|z|^2$ si $z \in \mb{Z}[\text{i}]$ ?

\item Montrer que les inversibles doivent être de module 1. En déduire l'ensemble des inversibles.


\end{enumerate}


\end{enumerate}



















\section{Fonctions réelles}

\subsection{Etude de fonctions}
        
\subsection{Fonctions circulaires réciproques}

\exercice

Justifier que $\arccos \dfrac{1-x^2}{1+x^2}$ est défini pour $x \in \mb{R}$, puis simplifier l'expression.

\exercice

Justifier que $\arcsin \dfrac{x}{\sqrt{x^2 +1}}$ est défini pour $x \in \mb{R}$, puis simplifier l'expression.




\exercice

Montrer que pour $a,b \in \mb{R}, ab \neq 1$ on a $\arctan(a) + \arctan(b) = \arctan \dfrac{a+b}{1-ab} + k \pi$ avec $k \in \{ -1, 0, 1 \}$.


\exercice[Formule de Machin]

\begin{enumerate} 

\item Soit $\theta = \arctan \frac{1}{5}$, calculer $\tan(2\theta)$, $\tan(4 \theta)$ et $\tan(4 \theta - \frac{\pi}{4})$.

\item Justifier que $\arctan(\tan(4 \theta-\frac{\pi}{4})) = 4 \theta-\frac{\pi}{4}$.

\item En déduire que $\displaystyle \frac{\pi}{4} = 4 \arctan\frac{1}{5} -  \arctan\frac{1}{239}$.

\end{enumerate}

\exercice 

Étudier et tracer la fonction $f : x \mapsto \arcsin \sin x + \arccos \cos x$.

\exercice 

Simplifier la fonction $f : x \mapsto \cos 2 \arccos x$




\cours 

\begin{enumerate}
    \item Donner le domaine de définition et tracer 
        la courbe de la fonction $x \mapsto b (x - a)^2 + a$

    \item Donner le domaine de définition et tracer 
        la courbe de la fonction $x \mapsto b \sqrt{x - a} + a$

    \item Donner le domaine de définition et tracer 
        la courbe de la fonction $x \mapsto \frac{x}{x^2 + 1}$


    \item Donner le domaine de définition et tracer la courbe 
        de $x \mapsto \sin(x) \sin (\alpha x)$ avec $\alpha$ très
        supérieur à $1$

    \item Donner la définition d'une bijection de $I \subseteq \mb{R}$
        dans $J \subseteq \mb{R}$

    \item Donner la définition d'une fonction croissante 
\end{enumerate}

\exercice[Fonction Bornée]

Soit $f : I \to J$ avec $I$ et $J$ des intervalles de $\mb{R}$.

\begin{enumerate}
    \item Montrer la formule suivante: $\forall x \in I, \exists M, f(x) \leq
        M$.

    \item Est-ce que $f$ est majorée sur $I$ ?
\end{enumerate}

\exercice[Calcul de dérivée]

On pose $f(x) = \frac{1}{1-x}$.

\begin{enumerate}
    \item Donner de domaine de définition de $f$
    \item Tracer la courbe de $f$
    \item Calculer la dérivée de $f$ en un point
    \item On note $f'$ la fonction dérivée de $f$,
        calculer la dérivée de $f'$
    \item En déduire une expression générale 
        de la dérivée $n$-ème de $f$
\end{enumerate}


\exercice[Dérivée et minimum local]

Soit $f$ une fonction dérivable sur un intervalle $]a,b[$ de $\mb{R}$,
et continue sur $[a,b]$.

\begin{enumerate}
    \item Si $f$ est croissante, que dire d'un éventuel minimum de $f$ ? 
        D'un maximum de $f$ ?
    \item Si $f$ admet un minimum en un point $x \in ]a,b[$, 
        que dire de $f'(x)$ ?
    \item Si $f$ admet un maximum en un point $x \in ]a,b[$,
        que dire de $f'(x)$ ?
    \item Si $f$ admet un extremum local en un point $x \in ]a,b[$
        que dire de $f'(x)$ ?
    \item Si $x \in ]a,b[$ et que $f'(x) = 0$, est-ce que $x$
        est un extremum ? Un extremum local ?
    \item Trouver un point réalisant un
        minimum pour la fonction $x \mapsto \sqrt{x}$ pour $x \geq 0$.
\end{enumerate}

\exercice[Convexité]

On dit que $f : I \to \mb{R}$ est convexe quand elle vérifie 
la propriété suivante

\begin{equation*}
    \forall (x,y) \in I^2, \forall \lambda \in [0,1], 
    f(\lambda x + (1-\lambda)y) \leq \lambda f(x) + (1-\lambda) f(y)
\end{equation*}

Soit $a \in I$, on note $\Delta_a$ la fonction définie comme suit

\begin{equation*}
    \Delta_a (x) = \frac{f(x) - f(a)}{x - a}
\end{equation*}

\begin{enumerate}
    \item Quel est l'ensemble de définitiion de $\Delta_a$ ?
    \item Si $(x,y) \in I^2$ et $\lambda \in [0,1]$ 
        montrer que $\lambda x + (1 - \lambda y) \in I$
    \item Montrer que si pour tout $a \in I$ $\Delta_a$ 
        est une fonction croissante alors $f$ est convexe.
        \emph{(Indication: supposer $x < y$, $\lambda \in ]0,1[$ 
            et remarquer 
            $\lambda x + (1 - \lambda y) < y$
            puis utiliser la croissance de $\Delta_x$)}
    
    \item On suppose maintenant $f$ convexe, et $a \in I$.
        \begin{enumerate}
            \item Si $(x,y) \in I^2$ et $x < y < a$, en considérant 
                $\lambda = (x - a) / (y-a)$ montrer que 
                $\Delta_a (x) \leq \Delta_a (y)$

            \item Si $(x,y) \in I^2$ et $a < x < y$, montrer que 
                $\Delta_a(x) \leq \Delta_a (y)$

            \item Si $(x,y) \in I^2$ et $x < a < y$, montrer 
                que $\Delta_a (x) \leq \Delta_a (y)$. 
                \emph{(Indication: considérer $\lambda = (a - y)/(x-y)$)}
        \end{enumerate}
        En déduire que $\Delta_a$ est croissante.
    \item Conclure que $f$ est convexe si et seulement si $\Delta_a$ est 
        croissante pour tout $a \in I$.
   
    \item En déduire que si $f$ est convexe et dérivable sur $I$, alors $f'$ 
        est croissante sur $I$ \emph{(Indication: utiliser la croissance de $\Delta_x$ 
        pour $x$ entre $a$ et $b$)}.

    \item Que dire de la réciproque ? \emph{(Indication: utiliser 
            $g(\lambda) = \lambda f(x) + (1 - \lambda)f(y) - f (\lambda x +
            (1-\lambda) f (y))$)}

    \item Montrer que si $f$ est deux fois dérivable, $f$ est convexe 
        si et seulement si $f''$ est positive.
\end{enumerate}


\exercice[Inégalité de Huygens]

\begin{enumerate}

\item Montrer que pour $x \in [0, \frac{\pi}{2}[$ on a $x \le \frac{2}{3} \sin(x) + \frac{1}{3} \tan(x) $.

\item Interpréter cette inégalité géométriquement.

\end{enumerate}


\exercice 

Soit $f : \mb{R} \rightarrow \mb{R}, x \mapsto \frac{2x}{1+x^2}$.
\begin{enumerate}
\item Etudier les variations de $f$. Tracer son graphe.

\item $f$ est-elle bijective ? (Surjective ? Injective ?)

\item Montrer que $g : [0,1] \rightarrow [0,1], x \mapsto f(x)$ est bijective.
\end{enumerate}




\exercice

On considère la fonction $f : \mb{R} \rightarrow \mb{R}, x \mapsto x^2 \sin(\frac{1}{x})$

\begin{enumerate}

\item Etudier le domaine de définition de $f$, sa parité, ses points d'annulation.

\item Montrer que $\forall x \neq 0$, $-x^2 \le f(x) \le x^2$. En déduire $\displaystyle \lim_{\substack{x \rightarrow 0\\x \neq 0}} f(x)$.

\item \begin{enumerate}

\item Montrer que pour tout $t\in [0,\frac{\pi}{2}]$, $\frac{2}{\pi}t \le \sin(t) \le t$.

\item En déduire un encadrement de $f(x)$ pour $x \ge a$ (on précisera la valeur de $a$).

\end{enumerate}

\item Tracer le graphe de $f$. On pourra admettre que $y =x$ est une asymptote de $f$ en $+ \infty$.
\end{enumerate}


\exercice[Argument sinus hyperbolique]

Résoudre l'équation $\text{sh}(x) =y \in \mb{R}$ d'inconnue $x \in \mb{R}$.


\exercice

\begin{enumerate}

\item Montrer que pour $x \neq 0$ on a $\dfrac{1}{\text{sh}(x)} = \dfrac{1}{\text{th}(x/2)} - \dfrac{1}{\text{th}(x)}$.


\item En déduire la limite de $\displaystyle \sum_{k=0}^n \frac{1}{\text{sh}(2^kx)}$ quand $n \rightarrow + \infty$.

\end{enumerate}

\exercice

\begin{enumerate}

    \item Soient $(a,b) \in \mb{R}^2$ et $f(x) = a \cos(x) + b \sin(x)$.
Trouver une condition nécessaire et suffisante sur $(a,b)$
pour qu'il existe un couple $(A,\Phi) \in \mb{R}^2$ tel que 
$f(x) = A \cos (x + \Phi)$

    \item Soient $(a,b) \in \mb{R}^2$ et $f(x) = a \cosh(x) + b \sinh(x)$.
Trouver une condition nécessaire et suffisante sur $(a,b)$
pour qu'il existe un couple $(A,\Phi) \in \mb{R}^2$ tel que 
$f(x) = A \cosh (x + \Phi)$
\end{enumerate}

\exercice

Soit $(x,y) \in \mb{R}^2$, montrer l'équivalence suivante 

\begin{equation*}
    x^2-y^2=1\text{ et }x>0
    \iff
    \exists! t \in \mb{R},
    \begin{cases}
        x = \cosh t \\
        y = \sinh t 
    \end{cases}
\end{equation*}

\exercice 

\begin{enumerate}

\item Calculer la dérivée $n$-ème de $x \mapsto \frac{1}{1+x}$, puis celle de $x \mapsto \frac{1}{1-x}$. On précisera les domaines de dérivabilité.

\item En déduire la dérivée $n$-ème de $x \mapsto \frac{1}{1-x^2}$ sur son domaine.

\end{enumerate}

\subsubsection{Dérivabilité}


\exercice[Formule de Leibniz et application]

\begin{enumerate}

\item Soient $f$ et $g$ deux fonctions réelles $n$ fois dérivables. Montrer par récurrence que $\displaystyle (fg)^{(n)} (x)= \sum_{k=0}^n {n \choose k} f^{(k)}(x) g^{(n-k)}(x)$.

\item \begin{enumerate}

\item Calculer la dérivée $n$-ème de $f:x \mapsto x^{2n}$.

\item En écrivant $x^{2n} = x^n\times x^n$, exprimer la dérivée $n$-ème de $f$ sous une autre forme.

\item En déduire que $\displaystyle \sum_{k=0}^n {n \choose k}^2 = {2n \choose n}$.

\end{enumerate}

\end{enumerate}

\exercice 

\begin{enumerate}

\item Donner un exemple \emph{simple} de fonction $f:[0,1] \rightarrow [0,1]$ vérifiant $\forall x, f(x^2) = f(x)$.
\item Soit $\displaystyle A = \left\{\text{e}^{-2^n\ln(2)}~\Big|~n \in \mb{Z}\right\}$.

\begin{enumerate}

\item Justifier que $A \subseteq[0,1]$.

\item Montrer que si $x \in A$, alors $x^2 \in A$.

\item Montrer que si $x \in [0,1]\smallsetminus A$, alors $x^2 \in [0,1]\smallsetminus A$.

\end{enumerate}

\item En déduire une fonction \emph{non constante} $f:[0,1] \rightarrow [0,1]$ vérifiant $\forall x, f(x^2) = f(x)$.
\end{enumerate}


\exercice

\begin{enumerate}

\item Trouver toutes les fonctions dérivables $f :\mb{R} \rightarrow \mb{R}$ vérifiant $\forall xy, f(x+y) = f(x)+ f(y)$.

\item Trouver toutes les fonctions dérivables $f :\mb{R} \rightarrow \mb{R}^*_+$ vérifiant $\forall xy, f(xy) = f(x)f(y)$.
\end{enumerate}



\exercice[Divergence de la série harmonique]

\begin{enumerate}

\item Calculer $\displaystyle \sum_{k=1}^n  \text{ln}(1+ \frac{1}{k})$.

\item Que peut-on en déduire quant à $\displaystyle \lim_{n \to + \infty} \sum_{k=1}^n \frac{1}{k}$ ? On utilisera une inégalité avec $\text{ln}$.
\end{enumerate}

\exercice

Résoudre le système $xy = 1$, ${\ex}^x {\ex}^y =a$ d'inconnues $x,y$, en fonction du paramètre $a \in \mb{R}$.


\exercice 

\begin{enumerate}
    \item Établir que pour $x \in ]0, +\infty[$ l'inégalité suivante
        est vérifiée
        \begin{equation*}
            \frac{1}{1+x} < \ln (1 + x) - \ln x < \frac{1}{x}
        \end{equation*}

    \item En déduire que pour $x \in ]0,+\infty[$ on a
        \begin{equation*}
            \left(1 + \frac{1}{x}\right)^x < e < \left(1 + \frac{1}{x}\right)^{1 + x}
        \end{equation*}
\end{enumerate}

\exercice 

\begin{enumerate}
    \item Établir que pour tout $x \geq 0$ et pour $p \in ]0,1[$ on a 
        l'inégalité suivante~:
        \begin{equation*}
            (1 + x)^p \leq 1 + x^p
        \end{equation*}

    \item En déduire que pour $x$ et $y$ réels, $p \in ]0,1[$
        l'inégalité suivante est vérifiée:
        \begin{equation*}
            (x + y)^p \leq x^p + y^p
        \end{equation*}
\end{enumerate}

\exercice

Étudier la fonction suivante et tracer sa courbe 

\begin{equation*}
    f : x \mapsto \left(1 + \frac{1}{x}\right)^x
\end{equation*}

\exercice

On pose $g(x) = e^{-1/x^2}$.

\begin{enumerate}
    \item Étudier le domaine de définition de $g$
    \item Montrer que l'on peut prolonger $g$ en $0$
    \item Montrer par récurrence sur $n$ l'équation suivante 
        (où $P_n$ est un polynôme dont on explicitera 
        le degré)

        \begin{equation*}
            g^{(n)} (x) = x^{-3n} P_n (x) f(x) 
        \end{equation*}

    \item Montrer que $f$ est $\mathcal{C}^\infty$, 
        et que toutes ses dérivées sont nulles en zéro

    \item Montrer que toutes les racines de $P_n$ sont 
        réelles \emph{(Indication: Roll'N'Roll)}
\end{enumerate}


\exercice 

Étudier la fonction $f : x \mapsto x \sqrt{1 - x^2}$
afin de tracer son graphe.

\exercice[Un petit théorème de point fixe]

Soit $f$ continue de $ [0,1] $ dans $[0,1]$, montrer qu'il existe 
$x \in [0,1]$ tel que $f(x) = x$.

\exercice

Étudier la fonction $f : x \mapsto \frac{\ln x }{x}$.


\exercice["Equations différentielles"]

\begin{enumerate}
    \item Trouver une fonction dérivable $f$ telle que $f'(x) = f(x)$
    \item Trouver une fonction dérivable $f$ telle que 
        $f'(x) = - f (x)$
    \item Trouver une fonction dérivable $f$ telle que 
        $f'(x) = -x f(x)$
\end{enumerate}

\section{Topologie, suites dans $\mb{R}$ et $\mb{C}$}


\exercice[Théorème de Knaster-Tarski]

Soit $f : [0,1] \rightarrow [0,1]$ une fonction croissante. On va montrer que $f$ possède un \emph{point fixe}, c'est-à-dire qu'il existe $a \in [0,1]$ tel que $f(a) = a$. Soit $A = \{x \in [0,1]~|~f(x)\le x\}$.

\begin{enumerate}

\item Montrer que $A$ n'est pas vide.

\item Montrer que si $x \in A$, alors $f(x) \in A$.

\item Soit $a = \inf A$. Montrer que $\forall x \in A, f(a) \le x$. En déduire que $f(a)\le a$.

\item Montrer que $a \le f(a)$ et conclure.

\end{enumerate}

\exercice

Soit $n \in \mb{N^*}$ et $x \in \mb{R}$. Démontrer que $\displaystyle \sum_{k=0}^{n-1} E\left(x + \frac{k}{n}\right) = E(nx)$ où  $E$ est la partie entière. \emph{Indication : faire un dessin de la droite réelle.}

\exercice

Soit $f : \mb{R} \rightarrow \mb{R}$ une fonction continue telle que si $ x \le y$ sont des rationnels alors $f(x) \le f(y)$. Montrer que $f$ est croissante sur $\mb{R}$.

\exercice

Soit $f : [0,1] \rightarrow \mb{R}$ continue et $\fonction{g}{[0,1]}{\mb{R}}{x}{\displaystyle \sup_{t \in [0,x]} f(t)}$.

Montrer que $g$ est croissante et continue.

\begin{enumerate}

\item Donner la définition d'une suite réelle convergente. Est-elle toujours bornée ?

\item Soient $(u_n)_{n \ge 0}$ une suite réelle telle que $(u_{2n}){n \ge 0}$ et $(u_{2n+1})_{n \ge 0}$ convergent vers la même limite $l$. Que dire de la convergence de $(u_n)_{n \ge 0}$ ?

\item \'Enoncer et démontrer le théorème de Bolzano-Weierstrass dans $\mb{C}$ (on pourra admettre sa validité dans $\mb{R}$).

\end{enumerate}

\exercice

Soient $(u_n)_{n \ge 0}$ et $(v_n)_{n \ge 0}$ deux suites de réels strictement positifs telles que $\frac{u_{n+1}}{u_n} \le \frac{v_{n+1}}{v_n}$.

\begin{enumerate}

\item On suppose que $(v_n)_{n \ge 0}$ tend vers $0$. Que dire de $(u_n)_{n \ge 0}$ ? \emph{Indication : télescopage.}

\item Que se passe-t-il si $(u_n)_{n \ge 0}$ tend vers $+ \infty$ ?

\end{enumerate}


\subsection{Convergence de suites}

\exercice[Règle de Leibniz]

\begin{enumerate}

\item Soit $(u_k)_{k \ge 0}$ une suite de réels positifs décroissante et tendant vers $0$. Pour $n \ge 0$ on pose $\displaystyle A_n := \sum_{k=0}^n (-1)^{k} u_k$.


\begin{enumerate}

\item Montrer que $(A_{2n})_{n \ge 0}$ et $(A_{2n+1})_{n \ge 0}$ sont adjacentes.

\item Montrer que $(A_n)_{n \ge 0}$ converge.

\end{enumerate}

\item Quelle est la nature de $\displaystyle \left(\sum_{k=0}^n \frac{(-1)^{k}}{\sqrt{k+1}}\right)_{n \ge 0}$ ?

\end{enumerate}





\exercice

\begin{enumerate}

\item Soit $k \in \mb{N}$ et $x \in \mb{R}$, montrer que que $nx -1 < \lfloor nx \rfloor \le nx$.

\item Que dire de $\displaystyle \lim_{n \to + \infty} \frac{1}{n^2} \sum_{k=1}^n \lfloor kx \rfloor$ ?
\end{enumerate}


\exercice[Suites convergentes vers les irrationnels]

\begin{enumerate}

\item Soit $(u_n)_{n \ge 0}$ une suite d'entiers convergente. Montrer qu'elle est stationnaire.

\item Soit $x\in \mb{R}^*_+ \smallsetminus \mb{Q}$ Monter qu'il existe une suite $\left(\frac{p_n}{q_n}\right)_{n \ge 0}$ de limite $x$ telle que $p_n, q_n \in \mb{N}^*$.

\item On suppose que $q_n \not \rightarrow + \infty$.

\begin{enumerate}

\item Montrer qu'il existe une sous-suite $(q_{\varphi(n)})_{n \ge 0}$ qui est bornée. En déduire qu'il existe une sous-suite $(q_{\psi(n)})_{n \ge 0}$ stationnaire.

\item Montrer que $\left(\frac{p_{\psi(n)}}{q_{\psi(n)}} \times q_{\psi(n)}\right)_{n \ge 0}$ est stationnaire.

\item Conclure à une contradiction.
\end{enumerate}

\item On sait maintenant que $q_n \to + \infty$. On suppose que $p_n \not \rightarrow + \infty$. Montrer qu'il existe une sous-suite $\left(\frac{p_{\nu(n)}}{q_{\nu(n)}}\right)_{n \ge 0}$ tendant vers $0$, puis conclure à une contradiction.

\end{enumerate}






\exercice[Complétude de $\mb{R}$]

Une suite réelle $(u_n)_{n \ge 0}$ est de Cauchy si $\forall \varepsilon >0$, $\exists n_0$, $\forall m,n \ge n_0$, $|u_n - u_m | \le \varepsilon$.

\begin{enumerate}

\item Soit $(u_n)_{n \ge 0}$ une suite réelle convergente. Montrer qu'elle est de Cauchy.

\item On suppose que $(u_n)_{n \ge 0}$ est de Cauchy.

\begin{enumerate}

\item Montrer que $(u_n)_{n \ge 0}$ est bornée.

\item En déduire qu'elle possède une sous-suite convergente vers une limite $l$. En réutilisant la propriété de Cauchy, montrer que $u_n \rightarrow l$.

\end{enumerate}

\item Ces résultats restent-ils vrais dans $\mb{C}$ ?

\end{enumerate}






\exercice[Lemme de Ces\`aro]

Soit $(u_k)_{k \ge 0}$ une suite réelle convergente vers une limite finie $l$. On souhaite montrer que la suite $\displaystyle S_n := \frac{1}{n+1} \sum_{k=0}^n u_k $ converge également vers $l$.

\begin{enumerate}

\item Montrer que pour tout $n \ge m \ge 0$ on a : 

$$ \left| \left( \frac{1}{n+1} \sum_{k=0}^n u_k \right) - l \right| \le \left(\frac{1}{n+1}  \sum_{k=0}^{m-1} \left| u_k - l \right|\right)+ \left(\frac{1}{n+1} \sum_{k=m}^n \left| u_k - l \right|\right)$$

\item On fixe à partir dans cette question et la suivante $\varepsilon > 0$.

\begin{enumerate}

\item Pourquoi existe-t-il $m \ge 0$ tel que $\forall k \ge m$, $|u_k - l | \le \varepsilon /2$ ?

\item En déduire que pour tout $n \ge m$, $\frac{1}{n+1} \sum_{k=m}^n \left| u_k - l \right| \le \varepsilon /2$.

\end{enumerate}

\item $m$ étant fixé, monter qu'il existe $n \ge m$ tel que $\frac{1}{n+1}  \sum_{k=0}^{m-1} \left| u_k - l \right| \le \varepsilon /2$.

\item Déduire de ce qui précède la convergence de $(S_n)_{n \ge 0}$.

\item A-t-on la réciproque de ce résultat ?




\end{enumerate}





\exercice[Extraction diagonale]

On suppose que pour tout $k \ge 0$, $u^k := (u^k_n)_{n \ge 0}$ est une suite réelle bornée. 

\begin{enumerate}

\item D'après quel théorème existe-t-il une sous-suite $\left(u^0_{\varphi_0(n)}\right)_{n \ge 0}$ convergente ?

\item Construire par récurrence une suite $(\varphi_k)_{k \ge 0}$ d'extractrices telle que :

\begin{itemize}

\item[$\bullet$] $\forall k \ge 0$, il existe une extractrice $\psi_k$ telle que $\varphi_{k+1} = \varphi_k \circ \psi_k$ ;

\item[$\bullet$] $\forall k \ge 0$, $\forall 1 \le i \le k$, $\left(u^i_{\varphi_k(n)}\right)_{n \ge 0}$ converge.

\end{itemize}

\item Montrer que $\varphi : k \mapsto \varphi_k(k)$ est une extractrice et que $\forall k \ge 0$, $\left(u^k_{\varphi(n)}\right)_{n \ge 0}$ converge.


\end{enumerate}


\exercice

Soit $n \in \mb{N^*}$ et $x \in \mb{R}$. Démontrer que $\displaystyle \sum_{k=0}^{n-1} \left\lfloor x + \frac{k}{n}\right\rfloor = \left\lfloor nx \right\rfloor$.

\emph{Indication : faire un dessin de la droite réelle.}


\section{Intégration}

\subsection{Primitives, intégrales}

\subsubsection{Calculs d'intégrales}

\exercice

Calculer $\displaystyle \int_{0}^{\frac{\pi}{4}} \cos^2(t) \di t$.

\exercice

Calculer $\displaystyle \int_{0}^{\frac{\pi}{8}} \tan(-2t) \di t$.

\exercice

Calculer $\displaystyle \int_{0}^{\sqrt{\ln(3)}} -t\ex^{t^2} \di t$.

\exercice

Calculer $\displaystyle \int_0^1 \ln(1+t^2) \di t$.


\exercice

Soit $\displaystyle I = \int_{0}^{\frac{\pi}{2}} \frac{\cos(t)}{\sqrt{1+ \cos(t) \sin(t)}} \di t$ et $\displaystyle I = \int_{0}^{\frac{\pi}{2}} \frac{\sin(t)}{\sqrt{1+ \cos(t) \sin(t)}} \di t$.

\begin{enumerate}

\item Montrer que $I = J$.

\item Calculer $I+J$ en posant $u = \sin(t) - \cos(t)$. En déduire $I$.

\end{enumerate}


\exercice
Montrer que $\displaystyle \int_{0}^{\sqrt{3}} \arcsin \frac{2t}{1+t^2} \di t = \frac{\pi}{\sqrt{3}}$.

\emph{Indication : on pourra intégrer par parties le facteur 1 multipliant l'arc.}


\exercice

Monter que $\displaystyle \int_{\frac{1}{2}}^2 \left(1+\frac{1}{t^2}\right) \arctan (t) \di t = \frac{3 \pi}{4}$.

\emph{Indication : commencer par poser $u = 1/t$}


\exercice[Une intégrale impropre]

Soit $a<b$ et $\varepsilon > 0$. Montrer que $\displaystyle \int_{a+ \varepsilon}^{b - \varepsilon} \frac{\di t}{\sqrt{(b-t)(t-a)}} = \int_{\varepsilon/(b-a)}^{1-\varepsilon/(b-a)} \frac{\di u}{\sqrt{u(1-u)}}$ puis calculer sa valeur. En faisant tendre $\varepsilon$ vers 0, en déduire que $\displaystyle \int_{a}^{b} \frac{\di t}{\sqrt{(b-t)(t-a)}}  = \pi$.

\exercice 

Calculer 

\begin{equation*}
    \int_0^1 \frac{1}{e^t + 1} dt 
\end{equation*}

\exercice 

Calculer 

\begin{equation*}
    \int_0^1 \sqrt{1 - t^2} dt 
\end{equation*}

\exercice 

Forme général d'une primitive de 
\begin{equation*}
    f(x) = \frac{1}{ax^2 + bx + c}
\end{equation*}

\exercice 

Calculer en utilisant la tangeante de l'arc moitié
\begin{equation*}
    \int_a^x \frac{dt}{\sin t}
\end{equation*}


\exercice

\begin{enumerate}
    \item Soit $P$ un polynôme de degré $n$ et 
        $f(x) = P(x) e^{\alpha x}$. Montrer qu'une 
        primitive de $f$ est de la forme $Q(x) e^{\alpha x}$
        avec $Q$ un polynôme de degré $n$

    \item Calculer une primitive de 
        la fonction $f(x) = 4x^3 e^x + 12x^2 e^x + 2x e^x + 3e^x$
\end{enumerate}

\exercice[Des IPP]

\begin{enumerate}
    \item Calculer une primitive de $f(x) = x \sin x$
    \item Calculer une primitive de $f(x) = x e^x$
    \item Calculer 
        \begin{equation*}
            \int_0^1 t^2 \sqrt{2t + 1}
        \end{equation*}
\end{enumerate}



\subsubsection{Suites d'intégrales}


\exercice[Wallis]

\exercice[Puissances de $\ln$]

Pour $n \ge 0$ soit $\displaystyle I_n := \int_{1}^{\ex} \ln(t)^n {\di} t$.

\begin{enumerate}

\item Calculer $I_0$ et $I_1$.

\item Montrer (sans calcul) que la suite $(I_n)_n$ possède une limite.

\item Etablir une relation de récurrence liant $I_{n+1}$ et $I_n$.

\item Montrer que $\forall n \ge 0$, $0<I_n<\frac{e}{n+1}$ et en déduire $\lim_n I_n$.

\item Montrer que $I_n = (-1)^n(a_n \ex - n!)$ où $(a_n)_{n \ge 0}$ est une suite d'entiers $>0$ vérifiant une relation de récurrence que l'on explicitera.

\end{enumerate}


\subsubsection{Intégrales fonction des bornes}


\exercice

Soit $ \fonction{f}{\mb{R^*_+}}{\mb{R}}{x}{\displaystyle \int_x^{2x} \frac{\ex^t}{t}}$.

\begin{enumerate}

\item Montrer que $f$ est dérivable et calculer sa dérivée.

\item En remarquant que pour $x>0$, $\forall t \in [x,2x]$, $\displaystyle \frac{\ex^x}{t} \le \frac{\ex^t}{t} \le \frac{\ex^{2x}}{t} $, déterminer $\displaystyle \lim_{x \rightarrow 0^+} f(x)$.

\end{enumerate}

\exercice

Soit $f : \mb{R} \rightarrow \mb{R}$ une fonction continue. Déterminer $\displaystyle \lim_{x \rightarrow 0^+} \frac{1}{x} \int_0^x f(t) \di t$.

\emph{Indication : introduire une primitive de $f$.}



\exercice[Convolution fonction des bornes]

Soit $g : \mb{R} \rightarrow \mb{R}$ une fonction continue. On pose $f(x) = \displaystyle \int_0^x \sin(x-t) g(t) \di t$.

\begin{enumerate}

\item Montrer que $f$ est dérivable et que $f'(x) =  \displaystyle \int_0^x \cos(x-t) g(t) \di t$.

\item Montrer que $f$ est deux fois dérivable et qu'elle vérifie $f'' + f = g(x)$.

\item Montrer que $f$ est de classe $\mc{C}^\infty$.

\end{enumerate}


\subsubsection{Applications}

\exercice[Comparaison série-intégrale]

\begin{enumerate}

\item Que vaut $\displaystyle \lim_{n \rightarrow + \infty} \int_1^{n} \frac{1}{t} \di  t$ ?

\item Si $k \ge 2$, montrer que $ \displaystyle \int_{k-1}^{k} \frac{1}{t} \di  t \le \frac{1}{k} \le \int_{k}^{k+1} \frac{1}{t} \di  t$.

\item En déduire $\displaystyle  \lim_{n \rightarrow + \infty} \sum_{k = 1}^{n} \frac{1}{k}$. Comparer avec $\displaystyle  \lim_{n \rightarrow + \infty} \sum_{k = 1}^{n} \frac{1}{2^k}$.
\end{enumerate}


\subsection{Equations différentielles}

\subsubsection{Linéaires d'ordre 1}

\exercice

Résoudre sur $\mb{R}$ l'équation différentielle $y' + 2y = x^2$.

\exercice

Résoudre sur $\mb{R}$ l'équation différentielle  $y' + y=2 \sin(x)$.


\exercice

Résoudre sur $\mb{R}$ l'équation différentielle $(x^2+1)y'+2xy+1= 1 + 3x^2 $

\exercice 

Résoudre l'équation $x y'(x) + \alpha y(x) = 0$ 


\exercice 

Résoudre l'équation $y'(x) + y(x) = 2 \sin (x)$

\exercice 

Résoudre l'équation $3 y'(x) + 5 y(x) = \arctan (x)$

\exercice[Équation Fonctionnelle]

Trouver toutes les applications $f : \mb{R} \to \mb{R}$ 
dérivables en $0$ telles que: 

\begin{equation*}
    f(x + y) = e^x f(y) + e^y f(x)
\end{equation*}

\emph{Indication: f(0) = 0, puis dériver en $x$ quelconque ...}

\exercice[Gronwall]

Soit $\psi > 0$ et $\phi$ deux fonctions vérifiant 
pour tout $t \geq t_0$~:

\begin{equation*}
    \phi (t) \leq K + \int_{t_0}^t \psi (s) \phi (s) ds
\end{equation*}

\begin{enumerate}
    \item Montrer que pour tout $t \geq t_0$~:
\begin{equation*}
    \phi (t) \leq K \exp \left(  \int_{t_0}^t \psi (s) ds \right) 
\end{equation*}

    \item Si on suppose de plus $K = 0$ et $\phi \geq 0$, 
        en déduire que $\phi = 0$
\end{enumerate}

\emph{Ceci permet de montrer dans certains cas l'unicité d'une solution}


\exercice[Périodicité]
Soit $\phi$ une fonction continue $T$-périodique sur $\mb{R}$.
Soit $a \in \mb{R}^*$. Déterminer toutes les solutions périodiques 
de l'équation $y' - ay = \phi$.

\exercice[Inégalité différentielle]

On travaille sur un intervalle fermé $I = [a,b]$.
Soit $\alpha : I \to \mb{R}^+$ et $f,g$ vérifiant~:

\begin{equation*}
    \begin{cases}
        f(a) \leq g(a) \\
        \forall t \in I, f'(t) \leq \alpha (t) f(t) \\
        \forall t \in I, g'(t)  = \alpha (t) g(t) \\
    \end{cases}
\end{equation*}

Montrer que pour tout $t \in I$ l'inégalité $f(t) \leq g(t)$ est vérifiée.


\subsubsection{Linéaires d'ordre 2}

\exercice

Résoudre sur $\mb{R}$ l'équation différentielle $y'' - 3y' + 2y = \sin(2t)$.

\exercice[Régime forcé]

Résoudre sur $\mb{R}$ l'équation différentielle $y'' + \omega^2 y = \cos(\omega_0 t)$.

\exercice 

Résoudre en utilisant la méthode d'abaissement 
de l'ordre ($y(x) = z(x) e^{r x}$) l'équation
suivante:

\begin{equation*}
    y''(x) + 4y(x) = \frac{1}{\cos 2x }
\end{equation*}

\exercice

Résoudre sur $\mb{R}$ l'équation différentielle  $(1-t^2)y' + t y = 0$.


\exercice
Trouver les fonctions $f : \mb{R} \rightarrow \mb{R}$ dérivables telles que $\forall x \in \mb{R}$, $f'(x) = f(2-x)$.

\emph{Indication : si $f$ est vérifie l'équation, elle est solution d'une ED d'ordre 2.}



\exercice

Trouver les fonctions $f : \mb{R} \rightarrow \mb{R}$ dérivables telles que $\forall x \in \mb{R}$, $f'(x) + f(-x) = \ex^x$.

\emph{Indication : si $f$ est vérifie l'équation, elle est solution d'une ED d'ordre 2.}



\exercice[Résolution générique]

Soit $g : \mb{R} \rightarrow \mb{R}$ une fonction continue. On cherche à résoudre $y'' + y = g$.

\begin{enumerate}

\item  On pose $\fonction{f}{\mb{R}}{\mb{R}}{x}{\displaystyle \int_0^x \sin(x-t) g(t) \di t}$.

\begin{enumerate}
\item Montrer que $f$ est dérivable et que $f'(x) =  \displaystyle \int_0^x \cos(x-t) g(t) \di t$.

\emph{Indication : on se souviendra que $\sin(x-t) = \cdots$.}

\item Montrer que $f$ est deux fois dérivable et qu'elle vérifie $f'' + f = g$.


\end{enumerate}

\item Résoudre $y'' + y = g$, en exprimant les solutions en fonction de $f$.

\item Application. Quelles sont les solutions si $g(t) = \cos(t) \sin(t)$ ? 

\end{enumerate}


\exercice[Equation d'Euler]

On cherche les $f : \mb{R}_+^* \rightarrow \mb{R}$ dérivables, telles que $f'(t) = f(1/t)$ pour tout $t>0$ $(E)$ .

\begin{enumerate}

\item Monter que si $f$ vérifie $(E)$, alors elle est infiniment dérivable.

\item Montrer que si $f$ vérifie $(E)$, alors $f$ est solution de $t^2 y'' + y = 0$ $(H)$.

\item Montrer que si $f$ est solution de $(H)$, alors $g := f \circ \exp$ vérifie une équation linéaire d'ordre 2 $(H')$ que l'on précisera.

\item Résoudre $(H')$ puis conclure.

\end{enumerate}

\subsubsection{Non-linéaires}

\exercice[Une équation de Bernoulli]

On cherche les solutions $\mb{R} \rightarrow \mb{R}_+^*$ de $y' = y +  \sqrt{y}$ $(B)$.

\begin{enumerate}

\item Montrer si $f$ est solution de $(B)$, alors $\sqrt{f}$ est solution de $2y' =  y + 1$ $(E)$.

\item Résoudre $(E)$ et conclure.

\end{enumerate}


\exercice[Une équation de Ricatti]

On cherche les solutions à valeurs complexes de $y' + y + y^2 + 1 = 0$ $(R)$. On raisonne par analyse-synthèse, soit $f$ une solution potentielle.

\begin{enumerate}

\item  \begin{enumerate}

\item A quelle condition sur $c \in \mb{C}$ la fonction constante $x \mapsto c$ est elle solution ?

\item Si $c \in \mb{C}$ est choisi comme à la question précédente, montrer que $g : t \mapsto f(t) -c$ est solution de $y' + (2\text{j} +1) y + y^2 = 0$ $(E)$

\end{enumerate}
 
\item \begin{enumerate}
\item On suppose que $g$ ne s'annule jamais. Montrer que $h : t \mapsto 1/g(t)$ est solution de $-y' + (2 \text{j}+1)y +1 = 0$.

\item Résoudre cette équation.

\end{enumerate}


\item Conclure quant à la forme des solutions de $(R)$ quand $g$ ne s'annule en aucun point.

\item Que se passait-il si $g$ s'annulait en un point dans l'équation $(E)$ ?

\emph{Indication : on pensera à Cauchy-Lipschitz.}
\end{enumerate}


\section{Espaces euclidiens}

\exercice[Quand la norme passe à l'intégrale]

Soit $E$ un espace euclidien de dimension finie et $f : [a,b] \rightarrow E$ une application continue telle que $\displaystyle \left\lVert\int_a^b f(t) \di t \right\lVert = \int_a^b \lVert f(t)\lVert \di t$.

\begin{enumerate}

\item Que dire si $\left\lVert\int_a^b f(t) \di t \right\lVert = 0$ ?

\item On suppose désormais $\neq 0$. Soit $e_1 = \int_a^b f(t) \di t \Big/  \left\lVert\int_a^b f(t) \di t \right\lVert $. On complète $e_1$ en une base orthonormée $e_1 \dots e_n$ de $E$, et on note $f = \sum_i f_i e_i$.
\begin{enumerate}

\item Montrer que $\int_a^b f(t) \di t = (\int_a^b \lVert f(t) \rVert \di t) e_1$ puis que $\int_a^b f(t) \di t = (\int_a^b \lVert f(t) \rVert \di t) e_1$.

\item Utiliser ce résultat pour obtenir $f_i = 0$ pour $i \ge 2$.

\item Conclure que $f = \lVert f \lVert e_1$.

\end{enumerate}

\item Ce résultat est-il toujours vrai avec une norme quelconque ?

\end{enumerate}





\end{document}
