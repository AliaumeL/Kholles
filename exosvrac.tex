\documentclass{article}
\usepackage[french]{babel}
\usepackage[T1]{fontenc}
\usepackage[utf8]{inputenc}
\usepackage{amsfonts, amsmath, amssymb, mathrsfs}
\usepackage [a4paper]{geometry}
\geometry{left=3.5cm, right=3.5cm,top=3.5cm ,bottom=4cm}
\usepackage{fancyhdr}
\usepackage{graphicx}
\usepackage{amsthm}
\usepackage{multicol}
\usepackage{complexity}
\usepackage{hyperref}
%\usepackage{enumitem}
\usepackage{paralist}
\usepackage{epigraph}
\usepackage{tikz}
\usetikzlibrary{arrows,positioning,automata,shadows}
\usepackage{palatino}
\usepackage{pdfpages}
\usepackage{epigraph}

%%%% Format

\setlength{\headheight}{35pt}
\lhead{Colles\\MPSI}
\rhead{}
\chead{}
\rfoot{}
\pagestyle{fancy}

\newcommand{\fonction}[5]{\raisebox{-0.5\baselineskip}{$\begin{array}{lccc}
    #1: & #2 & \longrightarrow & #3 \\
        & #4 & \longmapsto & #5 \end{array}$}}
        
\newcommand{\es}{\vspace{1\baselineskip}}
\newcommand{\ds}{\vspace{0.4\baselineskip}}

\newcommand{\mb}[1]{\mathbb{#1}}
\newcommand{\mc}[1]{\mathcal{#1}}
\newcommand{\mf}[1]{\mathfrak{#1}}

% Exercices

\newcounter{exo}
\setcounter{exo}{1}

\newcommand{\exercice}[1][\null]{\textbf{\\ \large Exercice \thesection.\theexo. \normalsize #1} \addtocounter{exo}{1}}
\newcommand{\cours}{\textbf{\ds \\ \large Questions de cours}}

%%%% Corps du fichier


\begin{document}

\tableofcontents

\pagebreak


\section{Logique, ensembles, applications}

\exercice 

Soit $f : \mb{R} \rightarrow \mb{R}$ une fonction. A-t-on toujours $\forall x \in \mb{R}, \exists y \in \mb{R}, f(x) = y$ ? Donner la négation de cette formule. Que dire des fonctions qui vérifient $\exists y \in \mb{R}, \forall x \in \mb{R},  f(x) = y$ ?

\exercice  Montrer que toute fonction $f : \mb{R} \mapsto \mb{R}$ s'écrit comme somme d'une fonction paire et d'une fonction impaire. Prouver l'unicité de cette écriture.

\exercice Donner les fonctions $f : \mb{N} \rightarrow \mb{N}$ telles que pour tous $a,b$ dans $\mb{N}$, $f(a+b) = f(a)+f(b)$.

\exercice  Calculer $\displaystyle \sum_{1 \le i, j \le n} ij$.




\exercice

Soient $A$ et $B$ deux parties de $E$. Montrer que $A \cap B = A \cup B$ implique $A = B$.



\exercice[Sous-structures]

On se donne une proposition $P(x)$ dépendante d'un réel $x$.

\begin{enumerate}

\item Soit $X \subseteq \mb{R}$ telle que $\forall x \in H, P(x)$. Soit $I \subseteq H$, montrer que $\forall x \in I, P(x)$.

\item Soit $J \subseteq \mb{R}$ telle que $\exists x \in H, P(x)$. Soit $J \subseteq K \subseteq \mb{R}$, montrer que $\exists x \in J, P(x)$.
\end{enumerate}

\exercice[Logique élémentaire]

\begin{enumerate}
    \item Prouver la formule suivante~: $\forall x \in \mathbb{R}, \exists y \in
        \mathbb{R}, x + y = 0$

    \item Donner la négation de cette formule. Si cette négation est vraie
        donner une preuve et si non donner un contre exemple explicite.

    \item Que dire de la formule suivante~: $\forall x \in \mathbb{R}, \exists y \in
        \mathbb{R}, xy = 1$ ?
\end{enumerate}


\exercice[Contre exemple à la distributivité des quantificateurs]

Trouver deux propositions $P(x)$ et $Q(x)$ telles que la propriété 
$(1)$ soit vraie, mais que la propriété $(2)$ soit fausse.

\begin{enumerate}[(1)]
    \item $\forall x \in \mathbb{R}, (P (x) \vee Q(x))$
    \item $\left(\forall x \in \mathbb{R}, P(x)\right) 
        \vee 
        \left(\forall x \in \mathbb{R}, Q(x)\right)$
\end{enumerate}

\exercice[Borne inférieure]

Montrer que $\forall x \in \mb{R}, \forall y \in \mb{R}, \left[ (\forall \varepsilon > 0, x \le y + \varepsilon) \Rightarrow (x \le y) \right]$.

\exercice

Trouver toutes les fonctions $f : \mb{Q}_{+}^* \rightarrow \mb{Q}_{+}^*$ vérifiant pour tous $a,b$, $f(ab) = f(a)f(b)$ et $f(a+b) = f(a)+f(b)$ .



\exercice

Soit $f : \mb{R} \rightarrow \mb{R}$ une fonction $T$-périodique pour tout $T \ge 0$. Montrer qu'elle est constante. (Une fonction $f$ est dite $T$-périodique si pour tout $x \in \mb{R}$, $f(x) = f(T+x)$).

\exercice

\begin{enumerate}

\item Montrer que $\sqrt{2} \not \in \mb{Q}$.

\item On veut montrer quet $\exists x, y \not \in \mb{Q}, x^y \in \mb{Q}$.

\begin{enumerate}

\item On suppose que $\sqrt{2}^{\sqrt{2}} \not \in \mb{Q}$. Montrer la propriété demandée.

\item Conclure.

\end{enumerate}



\end{enumerate}


\exercice[Sur les inclusions d'ensembles]

On pose $P = \{ 2k ~|~ k \in \mathbb{N} \}$ et $I = \{ 2k + 1 ~|~ k \in
\mathbb{N} \}$.

\begin{enumerate}
    \item Montrer par récurrence sur $n \in \mathbb{N}$ la propriété suivante~:
        \begin{equation}
            \forall n \in \mathbb{N}, \exists k \in \mathbb{N}, (n = 2k) \vee 
            (n = 2k + 1)
        \end{equation}
    \item En déduire une inclusion entre $P \cup I$ et $\mathbb{N}$
    \item Montrer l'inclusion réciproque
    \item En déduire que $P \cup I = \mathbb{N}$
    \item Montrer que $P \cap I = \emptyset$
    \item On dit qu'un entier est pair quand il est dans $P$ et qu'un 
        entier est impair quand il est dans $I$. Qu'avons nous montré 
        dans cet exercice ?
\end{enumerate}

\exercice[Croissance des fonctions]

On dit que $f : I \to \mb{R}$ est croissante 
quand elle vérifie la propriété suivante~:

\begin{equation}
    \forall (x,y) \in I^2, x \leq y \implies f(x) \leq f(y)
\end{equation}

On dit qu'une fonction $f : I \to \mb{R}$ est \emph{strictement}
croissante quand elle vérifie la propriété de croissance avec 
des inégalités strictes.

\begin{enumerate}
    \item Que dire si l'on remplace l'implication par une équivalence 
        dans la définition de croissance ?
    \item Que dire si l'on remplace l'implication par une équivalence 
        dans la définition de la croissance stricte ?
\end{enumerate}


\exercice[Finitude]

Soit $E$ un ensemble, montrer que les deux propositions suivantes 
sont équivalentes~:

\begin{enumerate}[(i)]
    \item $E$ est fini
    \item Toute fonction $f : E \to E$ admet une partie stable non 
        triviale
\end{enumerate}

Une partie stable est un sous ensemble $F \subseteq E$ tel 
que $f(F) \subseteq F$. Une partie triviale est une partie $F$
de $E$ qui est égale à $\emptyset$ ou $E$.



\exercice 
Soit $f : E \rightarrow E$. Soit $A \subseteq E$, on définit par récurrence la suite d'ensembles $(A_n)_{n \ge 0}$ par $A_0 = A$ puis $A_n = f(A_{n-1})$ pour $n \ge 1$. Soit $\displaystyle B = \bigcup_{n \in \mb{N}} A_n$.

\begin{enumerate}

\item Montrer que $f(B) \subseteq B$.

\item Soit $C \subseteq E$ telle que $A \subseteq C$ et $f(C) \subseteq C$. Montrer que $B \subseteq C$.
\end{enumerate}





\exercice 

Déterminer une bijection de $[-1,1]$ dans $\mb{R}$.


\section{Calcul algébriques, sommes, produits}


\exercice[Autour de la somme des $k$]

Notons $S_n = \sum_{k=1}^n k $ pour $n \ge 1$.

\begin{enumerate}

\item Donner la formule du cours (sans somme) pour $S_n$, puis la prouver par récurrence.

\item En calculant $2 S_n$, re-prouver ce résultat sans récurrence.

\item Re-re-prouver ce résultat en calculant $S_n = \sum_{k=1}^n (k+1)^2 -  \sum_{k=1}^n k^2  $ de deux manières différentes.
\end{enumerate}




\exercice[Inégalité faible de Jensen]

Soit $f: \mb{R} \mapsto \mb{R}$ une fonction vérifiant pour tous réels $x_1, x_2$ et $t \in [0,1]$
$$f(t x_1 +(1-t) x_2) \le t f(x_1) +(1-t) f(x_2)$$  (une telle fonction est dite convexe).
Montrer que si les $\lambda_1 \dots \lambda_n$ sont des réels positifs ou nuls tels que $\sum_{i=1}^n \lambda_i = 1$, alors pour tous réels $x_1 \dots x_n$ on a :
$$f\left(\sum_{i=1}^n \lambda_i x_i\right) \le \sum_{i=1}^n \lambda_i f(x_i).$$

\exercice[Inégalité de Bernoulli]

\begin{enumerate}

\item Montrer par récurrence que $(1+x)^n \ge 1+nx$ pour $n \ge 1$ et $x \ge -1$.

\item Prouver ce résultat sans récurrence (pour $x \ge 0$).

\end{enumerate}

\exercice[Série croissante (qui en fait n'est pas du tout une série :D)]

Montrer que la série $S_n$ définie ci-dessous est strictement croissante.

\begin{equation*}
    S_n = \sum_{k = 1}^n \frac{1}{n + k}
\end{equation*}

\exercice[Petites variations]

\begin{enumerate}
    \item Calculer $S_n = \sum_{k =1}^{n} k 2^{-k}$ de deux manières 
        différentes
    \item En déduire une expression de  $T_n = \sum_{k=1}^{n} k 2^{-k+1}$
\end{enumerate}

\exercice[Encadrement de Gauss]

On va donner un encadrement de $n !$.

\begin{enumerate}

\item Montrer que $\displaystyle n! = \prod_{\substack{i,j \ge 1\\i+j = n+1}}  \sqrt{ij}$

\item Montrer que pour tous $i,j \ge 1$ on a $i+j-1 \le ij \le (\frac{i+j}{2})^2$.

\item Conclure que $n^{\frac{n}{2}} \le n ! \le (\frac{n+1}{2})^n$.

\end{enumerate}




\exercice[Série harmonique]

Soit $ H_n := \sum_{k =1}^n 1/k$ (série harmonique).

\begin{enumerate}

\item Calculer $H_n$ pour $1\le n \le 5$.

\item Montrer que $H_n$ n'est jamais entier. On conjecturera, grâce à la question précédente, une certaine propriété à prouver par récurrence forte.

\end{enumerate}



\exercice Calculer les sommes suivantes pour $n \ge 0$ et $\alpha, \beta \in \mb{R}$.

\begin{enumerate}

\item $\displaystyle \sum_{k=0}^n \sin(\alpha + k\beta)$ et $\displaystyle \sum_{k=0}^n \cos(\alpha + k\beta)$. 

\item $\displaystyle \sum_{k=0}^n {k \choose n} \sin(\alpha + k\beta)$ et $\displaystyle \sum_{k=0}^n {k \choose n} \cos(\alpha + k\beta)$.

\item $\displaystyle \sum_{k=0}^n \text{sh}(\alpha + k\beta)$ et $\displaystyle \sum_{k=0}^n \text{ch}(\alpha + k\beta)$. 

\item $\displaystyle \sum_{k=0}^n {k \choose n} \text{sh}(\alpha + k\beta)$ et $\displaystyle \sum_{k=0}^n {k \choose n} \text{ch}(\alpha + k\beta)$.

\end{enumerate}

\exercice

Soit $\displaystyle S_1 = \sum_{\substack{k=0 \\ k \text{ pair}}}^n {n \choose k}$ et $\displaystyle S_2 = \sum_{\substack{k=0 \\ k \text{ impair}}}^n {n \choose k}$. Calculer $S_1$ et $S_2$.

\exercice[Formule d'inversion de Pascal]

\begin{enumerate}

\item Soit $n > 0$, calculer $\displaystyle \sum_{k=0}^n (-1)^k {k \choose n}$. Que dire si $n = 0$ ?

\item Soient $l \le k \le n$ des entiers, montrer que $\displaystyle {n \choose k} {k \choose l} = {n \choose l} {n-l \choose k-l} $.

\item Soit $(x_n)_{n \ge 0}$ une suite réelle, on pose $(y_k)_{k \ge 0}$ la suite définie par $\displaystyle y_k = \sum_{l=0}^k {k \choose l} x_l$.

Montrer que $\displaystyle x_n = \sum_{k=0}^n (-1)^{n-k} {n \choose k} y_k$.

\end{enumerate}


\exercice

Montrer que pour tout $n \ge 0$ on a $\displaystyle \sum_{k=1}^n \frac{(-1)^{k+1}}{k} {n \choose k} = \sum_{k=0}^n \frac{1}{k}$.

\exercice[Cauchy-Schwarz par Lagrange]

Soient $a_1 \dots a_n$ et $b_1 \dots b_n$ des réels.

\begin{enumerate}

\item Calculer $\displaystyle \left(\sum_{i=1}^n a_i \right)\left(\sum_{i=1}^n b_i \right)$

\item Montrer que $\displaystyle \left(\sum_{k=1}^n a_k^2\right)\left(\sum_{k=1}^n b_k^2\right) - \left(\sum_{k=1}^n a_k b_k\right)^2 = \sum_{1 \le i < j \le n} (a_i b_j - a_j b_i)^2$

(identité de Lagrange).

\item En déduire que $\displaystyle \left(\sum_{k=1}^n a_k b_k\right)^2 \le \left(\sum_{k=1}^n a_k^2\right)\left(\sum_{k=1}^n b_k^2\right)$

(inégalité de Cauchy-Schwarz).

\item En déduire que $ \displaystyle \left(\sum_{k=1}^n c_k\right)\left(\sum_{k=1}^n \frac{1}{c_k}\right) \ge n^2$ si $c_1 \dots c_k$ sont des réels strictement positifs.

\end{enumerate}

\exercice

\begin{enumerate}

\item Rappeler la formule reliant $\sin(2x)$ avec $\cos(x)$ et $\sin(x)$.

\item Soit $x \in ]0, \pi [$, simplifier $P =\displaystyle \prod_{k=0}^n \cos(2^k x)$ (on commencera par calculer $ \sin (x) \times P$).

\end{enumerate}


\section{Nombres complexes, trigonométrie}

\exercice

Soient $z$ et $z'$ deux nombres complexes de module $1$. Montrer que $\displaystyle \frac{z+z'}{1+zz'}$ est un réél (quand il est défini).

\exercice[Inégalité triangulaire généralisée]

\begin{enumerate}

\item Montrer que pour tout $n \ge 1$, pour tous $z_1 \dots z_n \in \mb{C}$, on a $\displaystyle \left|\sum_{k=1}^n z_k\right| \le \sum_{k=1}^n | z_k |$.

On pourra utiliser, sans la redémontrer, l'inégalité triangulaire du cours.

\item Rappeler dans quel cas $|z_1 + z_2| = |z_1| + |z_2|$.

\item Montrer que s'il existe $1 \le i,j \le n$, $|z_i + z_j| < |z_i| + |z_j|$, alors $ \left|\sum_{k=1}^n z_k\right| < \sum_{k=1}^n | z_k |$ 

\item Déduire de 2 et 3 à quelles conditions l'inégalité de la question 1 est une égalité. L'interpréter géométriquement.

\end{enumerate}

\exercice

Soit $Z \in \mb{C}$, résoudre dans $\mb{C}$ l'équation $\text{e}^z  = Z$ d'inconnue $Z$.


\exercice

Trouver toutes les fonctions $f : \mb{C} \rightarrow \mb{C}$ telles que $\forall z~\in \mb{C} = f(z) + \text{i} f(\overline{z}) = 2i$.

\exercice

Soit l'application $f : z \mapsto \frac{1}{1-z}$.

\begin{enumerate}

\item Exprimer $f(\text{e}^{\text{i}\theta})$ sous la forme $a + \text{i} b$.

\item Montrer que $f$ est une bijection de $\mb{U}$ vers une droite que l'on précisera.

\end{enumerate}

\exercice

Soit $n \in \mb{N^*}$ et $x \in \mb{R}$. Démontrer que $\displaystyle \sum_{k=0}^{n-1} E\left(x + \frac{k}{n}\right) = E(nx)$ où  $E$ est la partie entière. \emph{Indication : faire un dessin de la droite réelle.}



\exercice 
On note $\mathbb{U}_n$ l'ensemble des racines n-èmes de l'unité dans
$\mathbb{C}$.

Montrer que $\mathbb{U}_n = \{ e^{2ik\pi / n} ~|~ 0 \leq k \leq n - 1 \}$.



\exercice

On note $\mathbb{U}_n$ l'ensemble des racines n-èmes de l'unité dans
$\mathbb{C}$.
\begin{enumerate}
    \item Calculer $\sum_{ \omega \in \mathbb{U}_n} \omega$
    \item Calculer $\sum_{ \omega \in \mathbb{U}_n} | 1 - \omega |$
\end{enumerate}




\exercice

Soit $n \ge 1$. Trouver les $z \in \mb{C}$ tels que $(z-1)^n = (z+1)^n$.


\exercice

Calculer $\displaystyle P = \prod_{k=0}^n(z^k + \overline{z}^{k})$ en fonction du module et de l'argument de $z$.




\exercice

On rappelle que l'exponentielle complexe est définie par $$\text{e}^{a+\text{i}b} := \text{e}^{a} \text{e}^{\text{i}b} = \text{e}^{a} (\cos(b) + \text{i} \sin(b)).$$

\begin{enumerate}


\item Soit $f: \mb{C} \rightarrow \mb{C}, z \mapsto \frac{\text{e}^{\text{i}z}+\text{e}^{-\text{i}z}}{2}$ et $g: \mb{C} \rightarrow \mb{C}, z \mapsto \frac{\text{e}^{\text{i}z}-\text{e}^{-\text{i}z}}{2\text{i}}$ .

Que valent $f(x)$ et $g(x)$ quand $x \in \mb{R}$ ? De quelles fonctions réelles $f$ et $g$ sont-elles les "extensions" aux nombres complexes ?

\item Montrer que $f(z)^2 + g(z)^2 = 1$. Cela est-il cohérent avec la question précédente ?

\item Prouver la formule de Moivre complexe : $\text{e}^{\text{i}z} = f(z) + \text{i} g(z)$.

\end{enumerate}




\exercice


On rappelle que l'exponentielle complexe est définie par $$\text{e}^{a+\text{i}b} := \text{e}^{a} \text{e}^{\text{i}b} = \text{e}^{a} (\cos(b) + \text{i} \sin(b)).$$
\begin{enumerate}


\item Soit $f: \mb{C} \rightarrow \mb{C}, z \mapsto \frac{\text{e}^{z}+\text{e}^{-z}}{2}$ et $g: \mb{C} \rightarrow \mb{C}, z \mapsto \frac{\text{e}^{z}-\text{e}^{-z}}{2}$ .

Que dire $f(x)$ et $g(x)$ quand $x \in \mb{R}$ ?

\item Montrer que $f(z)^2 - g(z)^2 = 1$.

\item Calculer $f(\text{i} x)$ et  $\frac{g(\text{i} x}{\text{i}}$ quand $x \in \mb{R}$.

\end{enumerate}





\exercice[Autour des racines 3-èmes]

\begin{enumerate}

\item On note $\text{j} = \text{e}^{\frac{2 \text{i} \pi}{3}}$. Que valent $j^3$ et $j^4$ ? Les représenter sur le cercle unité.

\item Quelles sont les solutions dans $\mb{C}$ l'équation $1 + z + z^2 = 0$.

\item Soit $f : \mb{C}^3 \rightarrow \mb{C}$ définie par $f(z_1, z_2, z_3) = \left[z_1 + \text{j} z_2 + \text{j}^2 z_3\right]^3$. Montrer que pour tous $z_1, z_2, z_3 \in \mb{C}$ on a $f(z_1, z_2, z_3) = f(z_2, z_3, z_1) = f(z_3, z_1, z_2)$.

\end{enumerate}

\exercice[Un peu plus de $j$]

On rappelle que $j = e^{2i\pi /3}$.
On considère les trois somme suivantes~:

\begin{equation*}
    A_n = \sum_{k = 0 \wedge k \equiv 0 [3]}^{n} { n \choose k} 
\end{equation*}

\begin{equation*}
    B_n = \sum_{k = 0 \wedge k \equiv 1 [3]}^{n} { n \choose k} 
\end{equation*}

\begin{equation*}
    C_n = \sum_{k = 0 \wedge k \equiv 2 [3]}^{n} { n \choose k} 
\end{equation*}

\begin{enumerate}
    \item Calculer $S_n = \sum_{k = 0}^n { n \choose k } j^k$
    \item Montrer que $S_n = A_n + j B_n + j^2 C_n$
    \item En déduire que $\overline{S_n} = A_n + j^2 B_n + jC_n$
    \item Calculer $A_n + B_n + C_n$
    \item En dédure une expression de $A_n$, $B_n$ et $C_n$
\end{enumerate}


\exercice[Entiers de Gauss]

On considère l'ensemble $\mb{Z}[\text{i}] := \{a + \text{i}b~|~a, b \in \mb{Z}\}$.

\begin{enumerate}

\item Montrer que si $z, z' \in \mb{Z}[\text{i}]$, alors $zz' \in \mb{Z}[\text{i}]$ et $z+z' \in \mb{Z}[\text{i}]$.

\item On cherche les $z \in \mb{Z}[\text{i}] $ tels que $ \frac{1}{z} \in \mb{Z}[\text{i}]$. De tels éléments sont dits \emph{inversibles}.

\begin{enumerate}

\item Montrer que $\text{i}$ est inversible, mais pas de $1+ \text{i}$.

\item Que dire de $|z|^2$ si $z \in \mb{Z}[\text{i}]$ ?

\item Montrer que les inversibles doivent être de module 1. En déduire l'ensemble des inversibles.


\end{enumerate}


\end{enumerate}








\exercice[Racines primitives de l'unité]

\begin{enumerate}

\item Soit $z \in \mb{U}_n$, montrer que $\{z^k~|~k \ge 0\} \subseteq \mb{U}_n$.

\item On dit que $z \in \mb{U}_n$ est une racine \emph{primitive} dans $\mb{U}
_n$ si  $\{z^k~|~k \ge 0\} = \mb{U}_n$. Quelles sont les racines primitives dans  $\mb{U}_1$ ? $\mb{U}_2$? $\mb{U}_3$? $\mb{U}_4$ ?

\item Montrer que si $z \in \mb{U}_n$ alors $\{z^k~|~k \ge 0\} = \{z^k~|~0 \le k < n\}$.

\item  Soit $z \in \mb{U}_n$ telle qu'il existe $ 0 < k < n$ avec $z^k = 1$. Montrer que $z$ n'est pas primitive dans $\mb{U}_n$.
\emph{Indication : compter le nombre d'éléments de $\{z^k~|~0 \le k < n\}$.}

\item Supposons maintenant que $z \in \mb{U}_n$ n'est pas primitive dans $\mb{U}_n$.  \begin{enumerate} 

\item Montrer qu'il existe $0 \le i\neq j < n$ tels que $z^i = z^j$.
\item En déduire qu'il existe $0 < k < n$ tel que $z^k = 1$.

\end{enumerate}

\item Conclure que $z \in \mb{U}_n$ est primitive dans $\mb{U}_n$ si, et seulement si, $\forall k < n, z \not \in \mb{U}_k$.

\end{enumerate}



\exercice[Fonctions symétriques de 2 variables]

Soient $z, z' \in \mb{C}$ on note $S = z + z'$ et $P = z z'$.

\begin{enumerate}

\item Exprimer $z^2 + z'^2$ en fonction de $P$ et $S$. Que peut-on en déduire si $P$ et $S$ sont réels ?

\item L'écriture précédente est de la forme $\sum_{i=1}^n \lambda_i P^{j_i} S^{j'_i}$ avec $\lambda_i \in \mb{R}$ et $j_i, j'_i \in \mb{N}$. Montrer que $z^2 + z'$ ne peut pas s'écrire de cette manière. On pourra prendre $z = \text{i}$ et $z' = - \text{i}$.

\item Exprimer $z^2 z' + z'^2z $ en fonction de $P$ et $S$. En déduire un expression de $z^3 + z'^3$.

\item Que peut-on dire pour $z^n + z'^n$ ?

\end{enumerate}


\exercice[Homothéties dans le plan complexe]

On considère $f$ une fonction qui vérifie la propriété 
suivante~:

\begin{equation*}
    \forall (z,z') \in \mathbb{C}^2,
    \forall (\alpha, \beta) \in \mathbb{R}^2,
    f( \alpha z + \beta z') = \alpha f (z) + \beta f(z')
\end{equation*}

\begin{enumerate}
    \item Montrer que $f(0) = 0$
    \item Montrer que $f$ est entièrement déterminée 
        par $f(1)$ et $f(i)$

    \item On suppose désormais que $f$ vérifie de plus 
        la propriété suivante~: $\forall z \in \mathbb{C},
        \exists \lambda \in \mathbb{R}, f (z) = \lambda z$.
        Montrer que $f$ vérifie alors~: $\exists \lambda
        \in \mathbb{R}, \forall z \in \mathbb{C}, f(z) = \lambda z$.
\end{enumerate}


\exercice[Transformée de Fourier Rapide]

On note $\alpha_0, \dots, \alpha_{n-1}$ des nombres réels.
On pose $\omega = e^{2i\pi / n}$, et 
pour $0 \leq 1 \leq n - 1$~:

\begin{equation*}
    \beta_i = \sum_{k = 0}^{n-1} \alpha_k (\omega^i)^k
\end{equation*}

\begin{enumerate}
    \item 
        Calculer $\sum_{i = 0}^{n - 1} \beta_i (\omega^{-1})^i$
    \item 
        Calculer $\sum_{i = 0}^{n-1} \beta_i (\omega^{-k})^i$

    \item En déduire qu'il est possible de récupérer les coefficients 
        $\alpha_k$ à partir des coefficients $\beta_k$.
\end{enumerate}

\section{Nombres réels}


\exercice
Soit $f : [0,1] \rightarrow [0,1]$ une fonction croissante. On va montrer que $f$ possède un \emph{point fixe}, c'est-à-dire qu'il existe $a \in [0,1]$ tel que $f(a) = a$. Soit $A = \{x \in [0,1]~|~f(x)\le x\}$.

\begin{enumerate}

\item Montrer que $A$ n'est pas vide.

\item Montrer que si $x \in A$, alors $f(x) \in A$.

\item Soit $a = \inf A$. Montrer que $\forall x \in A, f(a) \le x$. En déduire que $f(a)\le a$.

\item Montrer que $a \le f(a)$ et conclure.

\end{enumerate}


\end{document}
